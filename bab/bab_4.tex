\chapter{PENUTUP}
\section{Kesimpulan}
\noindent
Berdasarkan hasil perancangan, pengujian, dan implementasi sistem
diperoleh esimpulan sebagai berikut:
\begin{enumerate}
  \item Model deteksi objek berbasis YOLOv12 telah berhasil dirancang
    dan dilatih untuk mengenali kontainer kimia dengan akurasi dan
    presisi lokalisasi yang tinggi. Evaluasi menggunakan metrik mAP,
    menunjukkan performa yang sangat baik
    dalam mendeteksi objek dan menentukan posisi \textit{bounding
    box} secara akurat.
  \item Model deteksi kecacatan berbasis CVAE terbukti efektif dalam
    mengidentifikasi anomali visual pada permukaan kontainer. Model
    dilatih hanya pada citra kontainer tanpa cacat dan berhasil
    merekonstruksi citra normal dengan tingkat error rendah.
    Sementara itu, kontainer cacat menghasilkan error rekonstruksi
    yang signifikan lebih tinggi. Ambang batas optimal sebesar
    0,007183 ditentukan melalui analisis kurva ROC dan menghasilkan
    akurasi 100\% pada 25 samepl uji.
  \item Sistem yang dirancang berhasil mengintegrasikan seluruh
    proses secara otomatis: mulai dari akuisisi citra oleh kamera,
    deteksi objek oleh YOLOv12, dan klasifikasi cacat oleh CVAE, hingga
    penyortiran fisik kontainer menggunakan lengan robot. Hal ini
    membuktikan kelayakan sistem inspeksi visual cerdas untuk
    aplikasi industri teris
\end{enumerate}

\vspace{1em}

\section{Saran}
\noindent
Sistem yang telah dibangun masih memiliki ruang untuk pengembangan
lebih lanjut. Beberapa saran untuk penelitian selanjutnya antara lain:
\begin{enumerate}
  \item Menerapkan pendekatan \textit{supervised learning} seperti
    CNN atau varian YOLOv12 untuk klasiifikasi cacat, model dapat
    dikembangkan untuk mengenali berbagai tipe kerusakan secara
    spesifik sehingga meningkatkan akurasi deteksi dilingkungan industri nyata.
  \item Meningkatkan fleksibilitas gerakan lengan robot dengan
    mengganti kontrol \textit{hard-coded} menjadi pendekatan berbasis
    \textit{inverse kinematics}. Dengan metode ini, sudut
    \textit{servo} dapat dihitung secara dinamis berdasarkan
    koordinat target (x, y, z), memungkinkan
    sistem beradaptasi secara \textit{real-time} terhadap posisi
    objek tanpa pemrograman ulang.
  \item Mengembangkan sistem akuisis citra 360 derajat. Prototipe
    saat ini hanya menggunakan 1 kamera dengan sudut pandang tetap,
    yang menyebabkan bagian permukaan kontainer tertentu tidak
    terjangkau. Penambahan kamera atau sistem rotasi objek dapat
    memungkinkan inspeksi menyeluruh terhadap seluruh permukaan
    kontainer, sehingga mendeteksi cacat tersembunyi secara lebih efektif
\end{enumerate}
