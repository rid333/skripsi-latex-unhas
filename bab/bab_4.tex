\chapter{KESIMPULAN}
\section{Kesimpulan}
Berdasarkan hasil perancangan, analisis, dan implementasi alat yang
telah digunakan, maka dapat diambil kesimpulan sebagai berikut:
\begin{enumerate}
  \item Model deteksi objek berbasis YOLO telah berhasil dirancang dan
    dilatih untuk mengenali kontainer kimia dengan akurasi dan
    presisi lokalisasi yang sangat tinggi. Kinerja model, yang
    dievaluasi menggunakan metrik \textit{mean Average Precision} (mAP),
    menunjukkan kemampuan yang sangat bagus dalam mendeteksi
    keberadaan objek dan menentukan \textit{bounding box} secara akurat.
  \item Model deteksi kecacatan menggunakan algoritma \textit{convolutional
    variational autoencoder} (CVAE) terbukti efektif dalam
    mengidentifikasi anomali visual pada permukaan kontainer. Model
    dilatih pada \textit{dataset} citra normal (tanpa cacat). Hasilnya, model
    mampu merekonstruksi citra normal dengan \textit{reconstruction
    error} yang rendah. Sebaliknya, kontainer cacat menghasilkan \textit{error}
    yang signifikan lebih tinggi. Penentuan \textit{threshold} optimal sebesar
    0.007183 dilakukan melalui analisis kurva ROC. \textit{Threshold} ini
    mampu mencapai tingkat akurasi 100\% pada 25 sampel uji.
  \item Sistem ini mampu menjalankan alur kerja otomasi secara
    menyeluruh: mulai dari akuisisi citra oleh kamera, identifikasi
    objek oleh YOLO, analisis kecacatan oleh CVAE, hingga penyortiran
    fisik kontainer oleh lengan robot ke area yang telah ditentukan
    (cacat atau normal). Dengan demikian, penelitian ini berhasil
    mendemonstrasikan kelayakan sebuah sistem inspeksi visual cerdas
    dan otomatis untuk penanganan kontainer di lingkungan industri terisolasi.
\end{enumerate}

\vspace{1em}

\section{Saran}
Alat yang telah dibuat memiliki potensi untuk dikembangkan lebih
lanjut. Berikut adalah beberapa saran untuk penelitian selanjutnya:
\begin{enumerate}
  \item Penelitian selanjutnya disarankan mencoba pendekatan
    \textit{supervised learning}, seperti \textit{convolutional
    neural network} (CNN)
    atau YOLO yang diadaptasi untuk klasifikasi cacat. Dengan adanya
    label pada data cacat, model diharapkan dapat mengenali berbagai
    jenis kerusakan secara lebih spesifik dan meningkatkan akurasi
    deteksi pada kondisi industri nyata.
  \item Kontrol pergerakan lengan robot pada prototipe ini masih
    bersifat \textit{hard-coded}, dengan sudut \textit{servo} yang ditentukan
    sebelumnya. Untuk meningkatkan fleksibilitas, disarankan
    menggunakan \textit{inverse kinematics}, sehingga lengan dapat secara
    dinamis menghitung sudut sendi untuk mencapai koordinat target
    (x, y, z). Dengan demikian, sistem mampu beradaptasi \textit{real-time}
    terhadap perubahan posisi objek tanpa perlu pemrograman ulang manual.
  \item Prototipe ini memanfaatkan satu kamera dengan sudut pandang
    tunggal untuk akuisisi citra. Keterbatasan ini menyebabkan
    inspeksi hanya dapat dilakukan pada permukaan yang terlihat oleh
    kamera, sehingga potensi adanya cacat di sisi lain kontainer akan
    terlewatkan. Untuk mengatasi hal ini, penelitian selanjutnya
    dapat berfokus pada pengembangan sistem inspeksi 360 derajat.
\end{enumerate}
