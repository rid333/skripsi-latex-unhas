\chapter{Pendahuluan}
\pagenumbering{arabic}
\section{Latar Belakang}
Seiring dengan pesatnya perkembangan Artificial Intelligence (AI),
robotika, dan otomatisasi, penggunaan lengan robot otomatis semakin
meluas di sektor manufaktur, logistik, dan layanan [1]. Otomatisasi
bertujuan untuk menjalankan rangkaian tindakan sesuai dengan proses
yang telah ditetapkan tanpa intervensi manusia, dengan mengendalikan
perangkat mekanis secara otomatis [2]. Lengan robot mampu melakukan
berbagai tugas, seperti perakitan, penanganan, dan pengemasan,
sehingga manusia tidak perlu lagi melakukan pekerjaan yang
berulang-ulang [3]. Dalam industri manufaktur, lengan robot sering
dipadukan dengan sensor kamera, algoritma visi komputer, dan
teknologi otomatisasi guna mendeteksi cacat pada objek dengan tingkat
presisi dan efisiensi yang tinggi. \par

Dalam konteks industri, terutama pada penanganan material berbahaya
seperti kontainer kimia, aspek keamanan menjadi prioritas utama.
Inspeksi visual otomatis menggunakan robot diperlukan untuk
mengurangi risiko paparan zat berbahaya kepada manusia. Selain itu,
isolasi dalam proses deteksi cacat pada kontainer kimia sangat
penting, karena kerusakan atau kontaminasi pada kontainer dapat
menimbulkan risiko keselamatan bagi konsumen. Tingkat kesalahan
inspeksi manual juga cukup tinggi, berkisar 20\% hingga 30\%.
Beberapa faktor yang menyebabkan kesalahan ini, termasuk kelelahan,
stres, kesendirian, pencahayaan yang tidak memadai, dan kurangnya
pengalaman [4]. Peningkatan penggunaan robot juga meningkatkan
produktivitas tenaga kerja sebesar 0,36\% per tahun, meningkatkan
produktivitas total, dan menurunkan harga output [5]. Penelitian juga
menunjukkan, antara tahun 2005 dan 2011, jumlah robot per 1.000
pekerja meningkat sebesar 25\%, yang menyebabkan penurunan 72.658
cedera kerja per tahun. Penurunan cedera ini diperkirakan menghemat
\$1,67 miliar (sekitar Rp27,3 triliun) per tahun, atau total \$11,69
miliar (sekitar Rp191,2 triliun) selama periode tersebut [6]. Oleh
karena itu, penerapan metode otomatis yang tepat untuk mendeteksi
cacat dan kontaminan sangat krusial guna menjamin keamanan, mencapai
skalabilitas, dan meningkatkan efisiensi biaya.

Algoritma deep learning berupa Autoencoder dapat dimanfaatkan untuk
mendeteksi anomali dan cacat permukaan [7]. Model autoencoder,
khususnya Convolutional Autoencoder (CAE), dapat dilatih hanya dengan
sampel bebas cacat [8]. CAE bekerja dengan meminimalkan kesalahan
rekonstruksi sehingga fitur-fitur representatif dapat teridentifikasi
secara optimal. Dengan melatih model pada data normal, setiap sampel
yang mengandung cacat pada tahap evaluasi akan menunjukkan perbedaan
yang signifikan dibandingkan dengan data bebas cacat yang telah
dilatih [9]. Pendekatan ini sangat berguna ketika pengumpulan data
cacat terbukti sulit atau mahal. \par

Berbagai studi telah memanfaatkan YOLO bersama lengan robot untuk
beragam aplikasi. Liang Wang (2024) mengembangkan model kontrol
lengan robot berbasis YOLO yang lebih efisien untuk menggenggam
komponen logam secara akurat di lingkungan industri yang kompleks
[10]. Hirohisa Kato dkk. (2022) mengusulkan metode deteksi objek
parsial dan estimasi kedalaman menggunakan kombinasi YOLO dan CNN
untuk mengontrol lengan robot, yang terbukti efektif dalam eksperimen
pengambilan objek dengan robot 4DOF [11]. Munhyeong Kim (2021)
menerapkan YOLO dalam metode genggam robot untuk deteksi dan
pemilahan sampah secara real-time, dengan pendekatan pembatasan area
pasca deteksi objek [12]. Tantan Jin (2024) mengembangkan versi
peningkatan dari YOLOv8n dengan integrasi modul dilated
re-parameterization, feature pyramid, dan Scylla-IoU loss untuk
meningkatkan akurasi dan adaptabilitas lengan robot pemetik apel
dalam kondisi kebun yang kompleks [13]. \par

Sementara itu, Autoencoder adalah model jaringan saraf dalam yang
dilatih untuk merekonstruksi data input, dan digunakan dalam deteksi
cacat dengan membandingkan perbedaan signifikan antara input normal
dan hasil rekonstruksi dari data yang mengandung cacat. Andrea Bionda
(2022) menerapkan autoencoder dengan fungsi loss berbasis CW-SSIM
untuk mendeteksi anomali pada citra industri, yang terbukti lebih
efisien dibandingkan jaringan saraf besar lainnya [14]. Huitaek Yun
(2021) mengusulkan pendekatan autoencoder untuk mendeteksi kerusakan
pada lengan robot industri berdasarkan sinyal suara internal, dengan
memanfaatkan citra spektrogram STFT [15]. Haoyang Jia (2022)
memperkenalkan model deteksi anomali dengan struktur
encoder-decoder-encoder (EDE) dan pelatihan dua tahap, yang
menggabungkan rekonstruksi dan pendekatan konfrontasi generatif [16].
Nejc Kozamernik (2025) mengembangkan FuseDecode Autoencoder, yang
menerapkan pembelajaran bertahap mulai dari tanpa supervisi,
semi-supervisi, hingga supervisi campuran, dan menunjukkan keunggulan
dalam deteksi cacat pada data industri nyata serta dataset MVTec AD
[17]. Patrick Ruediger-Flore (2024) membandingkan klasifikasi biner
dan autoencoder dalam deteksi anomali citra untuk proses perakitan
rangka, khususnya pada kesalahan posisi dan rotasi komponen, dan
menyimpulkan bahwa autoencoder lebih unggul dalam mendeteksi anomali
halus serta fleksibel terhadap data terbatas [18]. \par

Kebutuhan untuk meningkatkan efisiensi dan keamanan dalam inspeksi
industri mendorong pemanfaatan AI dan robotika. Deteksi cacat secara
otomatis, terutama dalam penanganan material berbahaya, sangat
penting untuk memastikan kualitas produk dan keselamatan manusia.
Autoencoder merupakan algoritma yang efektif karena kemampuannya
untuk belajar dari data normal sekaligus mengidentifikasi anomali.
Dengan demikian, penelitian ini bertujuan untuk mengeksplorasi dan
mengoptimalkan penerapan autoencoder dalam mendeteksi cacat
permukaan, sehingga dapat berkontribusi pada kemajuan manufaktur yang
cerdas dan berkelanjutan.

\vspace{1em}

\section{Tujuan Penelitian}
Tujuan yang ingin didapatkan dalam penelitian yaitu:
\begin{enumerate}
  \item Merancang dan melatih model deteksi objek berbasis YOLO yang
    mampu mengenali kontainer kimia secara akurat dalam berbagai kondisi.
  \item Membangun model deteksi kecacatan menggunakan algoritma
    Convolutional Autoencoder yang efektif dalam membedakan antara
    kontainer cacat dan tidak cacat.
  \item Mengintegrasikan model deteksi objek dan deteksi kecacatan ke
    dalam sistem berbasis lengan robot untuk proses identifikasi dan
    penyortiran otomatis.
\end{enumerate}

\vspace{1em}

\section{Manfaat Penelitian}
Manfaat yang ingin didapatkan dalam penelitian yaitu:
\begin{enumerate}
  \item Memberikan kontribusi terhadap pengembangan sistem deteksi
    objek yang cepat dan akurat di lingkungan industri, khususnya
    untuk aplikas i inspeksi visual kontainer kimia dalam kondisi
    nyata yang bervariasi.
  \item Menyediakan solusi efektif dalam mendeteksi kecacatan secara
    otomatis, yang dapat menggantikan metode inspeksi manual yang
    memakan waktu dan rentan terhadap kesalahan manusia (human error).
  \item Mendorong otomatisasi penuh dalam proses identifikasi dan
    penyortiran kontainer kimia, sehingga meningkatkan efisiensi
    produksi, menurunkan biaya operasional, dan memperkecil risiko
    kesalahan klasifikasi dalam sistem manufaktur.
\end{enumerate}
