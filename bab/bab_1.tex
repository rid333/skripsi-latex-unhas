\chapter{PENDAHULUAN}
\pagenumbering{arabic}
\section{Latar Belakang}
\noindent
Seiring dengan pesatnya perkembangan AI,
robotika, dan otomatisasi, penggunaan lengan robot otomatis semakin
meluas di sektor manufaktur, logistik, dan layanan \citep{1}. Otomatisasi
bertujuan untuk menjalankan rangkaian tindakan sesuai dengan proses
yang telah ditetapkan tanpa intervensi manusia dengan mengendalikan
perangkat mekanis secara otomatis \citep{2}. Lengan robot mampu
melakukan berbagai tugas seperti perakitan, penanganan, dan
pengemasan sehingga dapat menggantikan pekerjaan yang bersifat
berulang dan melelahkan bagi manusia \citep{3}. Dalam industri
manufaktur, lengan robot sering
dikombinasikan dengan sensor kamera, algoritma visi komputer, dan
teknologi otomatisasi guna mendeteksi cacat pada objek dengan tingkat
presisi dan efisiensi yang tinggi.

Dalam konteks industri, terutama pada penanganan material berbahaya
seperti kontainer kimia, aspek keselamatan menjadi prioritas utama.
Inspeksi visual otomatis dengan bantuan robot diperlukan untuk
mengurangi risiko paparan zat berbahaya terhadap manusia. Isolasi
dalam proses deteksi cacat pada kontainer kimia sangat
penting, karena kerusakan atau kontaminasi pada kontainer dapat
menimbulkan ancaman keselamatan bagi konsumen. \citet{4} melaporkan
bahwa kesalahan
inspeksi manual berkisar 20\% hingga 30\% yang disebabkan oleh
beberapa faktor seperti kelelahan,
stres, pencahayaan yang tidak memadai, dan kurangnya
pengalaman. Peningkatan penggunaan robot dalam industri juga
berdampak positif terhadap produktivitas. \citet{5} menemukan bahwa
otomatisasi meningkatkan produktivitas tenaga kerja sebesar 0,36\%
per tahun, serta menurunkan harga produksi. \citet{6}
melaporkan bahwa antara 2005 dan 2011, jumlah robot per 1.000
pekerja meningkat sebesar 25\%, yang berkontribusi terhadap penurunan
72.658 kasus cedera kerja per tahun dengan potensi penghematan biaya
mencapai sekitar Rp27,3 triliun per tahun, atau total sekitar Rp191,2
triliun selama periode tersebut. Oleh
karena itu, penerapan metode otomatis yang tepat dalam mendeteksi
cacat dan kontaminasi sangat krusial untuk menjamin keselamatan
kerja, mencapai skalabilitas sistem, dan meningkatkan efisiensi biaya.

Mayoritas sistem model deteksi cacat saat ini menggunakan metode
\textit{supervised learning} seperti algoritma \textit{You Only Look
Once} (YOLO). \citet{29} mengusulkan model berbasis YOLOv8 dengan
integrasi modul \textit{Multi-Path Convolution Attention} (MPCA) dan
\textit{Partial Self-Attention} (PSA) untuk meningkatkan akurasi dan
sensitivitas dalam mendeteksi cacat permukaan baja yang telah diuji pada
\textit{dataset} publik \textit{Northeastern University-Defect Detection}
(NEU-DET) dan VOC2007. \citet{30} mengembangkan model YOLO untuk
mendeteksi cacat
pada komponen \textit{magnet tile motor} kendaraan listrik dengan cepat dan
akurat. \citet{31} juga mengusulkan model YOLO yang dioptimalkan
untuk mendeteksi cacat kecil pada pengecoran logam dengan
menggunakan modul deteksi objek kecil, fitur re-ekstraksi, dan
\textit{multi-scale attention}, serta diuji menggunakan \textit{dataset}
NEU-DET. Semua pendekatan tersebut menggunakan \textit{dataset}
berlabel, sehingga memerlukan data cacat yang sudah ditandai secara
manual sebagai prasyarat pelatihan.

Sebagai alternatif, algoritma \textit{deep learning} berupa
\textit{autoencoder} dapat dimanfaatkan untuk mendeteksi anomali dan
cacat permukaan tanpa memerlukan \textit{dataset} berlabel \citep{7}.
Model \textit{autoencoder}, khususnya \textit{convolutional
autoencoder} (CAE), dapat dilatih hanya dengan sampel data bebas cacat
\citep{8}. CAE bekerja dengan meminimalkan kesalahan rekonstruksi
sehingga fitur-fitur representatif dari objek dapat teridentifikasi
secara optimal. Karena model hanya dilatih menggunakan data normal,
setiap sampel yang mengandung cacat akan menunjukkan deviasi
signifikan saat proses evaluasi \citep{9}. Pendekatan ini sangat
berguna ketika pengumpulan data sulit atau mahal untuk dikumpulkan.

YOLO juga telah banyak digunakan bersama lengan robot dalam berbagai
aplikasi. \citet{10} mengembangkan sistem kontrol
lengan robot berbasis YOLO yang efisien untuk menggenggam
komponen logam secara presisi di lingkungan industri yang kompleks.
\citet{11} mengusulkan metode yang menggabungkan YOLO dan
\textit{Convolutional Neural Network} (CNN) untuk mendeteksi objek
parsial dan memperkirakan kedalaman serta mengontrol lengan robot
dengan 4 derajat kebebasan. \citet{12} menerapkan YOLO dalam metode
genggam robot untuk deteksi dan pemilahan sampah secara
\textit{real-time}, dengan pembatasan area pasca deteksi
objek. \citet{13} mengembangkan YOLOv8n yang
telah ditingkatkan dengan integrasi modul \textit{dilated
re-parameterization}, \textit{feature pyramid}, dan
\textit{Scylla-IoU loss} yang terbukti meningkatkan akurasi dan
adaptabilitas lengan robot dalam memetik apel di kebun yang kompleks.

Sementara itu, \textit{autoencoder} sebagai model jaringan saraf
dilatih untuk merekonstruksi data \textit{input} dan digunakan
dalam deteksi cacat dengan membandingkan perbedaan signifikan antara
\textit{input} normal dan hasil rekonstruksi.
\citet{14} menggunakan \textit{autoencoder} dengan
fungsi \textit{loss} berbasis \textit{Complex Wavelet Structural
Similarity} (CW-SSIM) untuk mendeteksi anomali citra industri, yang
terbukti efisien dibandingkan model \textit{deep learning} lainnya.
\citet{15} mengembangkan \textit{autoencoder} untuk mendeteksi kerusakan
pada lengan robot berdasarkan sinyal suara internal, menggunakan
citra spektrogram \textit{Short-Time Fourier Transform}
(STFT). \citet{16} mengusulkan model
deteksi anomali dengan arsitektur \textit{encoder-decoder-encoder}
(EDE) dan pelatihan dua tahap, yang
menggabungkan pendekatan rekonstruksi dan konfrontasi generatif.
\citet{17} mengembangkan \textit{FuseDecode} \textit{autoencoder},
dengan strategi pembelajaran bertahap mulai dari tanpa supervisi,
semi-supervisi, dan supervisi campuran, serta menunjukkan kinerja unggul
dalam deteksi cacat pada data industri nyata dan \textit{dataset}
MVTec \textit{Anomaly Detection} (AD). \citet{18} membandingkan
klasifikasi biner dan \textit{autoencoder} dalam deteksi anomali
citra untuk proses perakitan rangka, khususnya pada kesalahan posisi
dan rotasi komponen, dan menyimpulkan bahwa \textit{autoencoder}
lebih efektif untuk deteksi anomali halus dan fleksibel dalam kondisi
data yang terbatas.

Kebutuhan untuk meningkatkan efisiensi dan keselamatan dalam proses inspeksi
industri mendorong pemanfaatan AI dan robotika. Deteksi cacat
otomatis, terutama dalam penanganan material berbahaya, sangat
penting untuk menjamin kualitas produk dan keselamatan manusia.
\textit{Autoencoder} merupakan algoritma pembelajaran mesin yang
efektif karena kemampuannya untuk belajar dari data normal dan
mengidentifikasi anomalis secara mandiri.
Dengan demikian, penelitian ini bertujuan untuk mengeksplorasi dan
mengoptimalkan penerapan \textit{autoencoder} dalam mendeteksi cacat
permukaan, sehingga dapat mendukung pengembangan manufaktur
cerdas yang efisien dan berkelanjutan.

\vspace{1em}

\Needspace{3\baselineskip}
\section{Tujuan Penelitian}
\noindent
Penelitian ini bertujuan untuk:
\begin{enumerate}
  \item Merancang dan melatih model deteksi objek berbasis YOLOv12 yang
    mampu mengenali kontainer kimia secara akurat dalam berbagai lingkungan.
  \item Membangun model deteksi kecacatan menggunakan algoritma
    \textit{convolutional autoencoder} yang efektif dalam membedakan
    antara kontainer cacat dan tidak.
  \item Mengintegrasikan model deteksi objek dan deteksi kecacatan ke
    dalam sistem berbasis lengan robot untuk proses identifikasi dan
    penyortiran otomatis.
\end{enumerate}

\vspace{1em}

\section{Manfaat Penelitian}
\noindent
Manfaat yang diharapkan dari penelitian ini antara lain:
\begin{enumerate}
  \item Memberikan kontribusi terhadap pengembangan sistem deteksi
    objek yang cepat dan akurat di lingkungan industri, khususnya
    untuk inspeksi visual kontainer kimia dalam kondisi nyata yang bervariasi.
  \item Menyediakan solusi yang efektif dalam mendeteksi kecacatan secara
    otomatis, sehingga dapat menggantikan metode inspeksi manual yang
    memakan waktu dan rentan terhadap kesalahan manusia.
  \item Mendorong otomatisasi penuh dalam proses identifikasi dan
    penyortiran kontainer kimia, sehingga meningkatkan efisiensi
    produksi, menurunkan biaya operasional, dan meminimalkan risiko
    kesalahan klasifikasi dalam sistem manufaktur.
\end{enumerate}
