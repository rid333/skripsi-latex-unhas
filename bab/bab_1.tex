\chapter{PENDAHULUAN}
\pagenumbering{arabic}
\section{Latar Belakang}
Seiring dengan pesatnya perkembangan \textit{Artificial Intelligence} (AI),
robotika, dan otomatisasi, penggunaan lengan robot otomatis semakin
meluas di sektor manufaktur, logistik, dan layanan \citep{1}. Otomatisasi
bertujuan untuk menjalankan rangkaian tindakan sesuai dengan proses
yang telah ditetapkan tanpa intervensi manusia, dengan mengendalikan
perangkat mekanis secara otomatis \citep{2}. Lengan robot mampu melakukan
berbagai tugas, seperti perakitan, penanganan, dan pengemasan,
sehingga manusia tidak perlu lagi melakukan pekerjaan yang
berulang-ulang \citep{3}. Dalam industri manufaktur, lengan robot sering
dipadukan dengan sensor kamera, algoritma visi komputer, dan
teknologi otomatisasi guna mendeteksi cacat pada objek dengan tingkat
presisi dan efisiensi yang tinggi.

Dalam konteks industri, terutama pada penanganan material berbahaya
seperti kontainer kimia, aspek keamanan menjadi prioritas utama.
Inspeksi visual otomatis menggunakan robot diperlukan untuk
mengurangi risiko paparan zat berbahaya kepada manusia. Selain itu,
isolasi dalam proses deteksi cacat pada kontainer kimia sangat
penting karena kerusakan atau kontaminasi pada kontainer dapat
menimbulkan risiko keselamatan bagi konsumen. \citet{4} menyebut
tingkat kesalahan
inspeksi manual juga cukup tinggi, berkisar 20\% hingga 30\%.
Beberapa faktor yang menyebabkan kesalahan ini, termasuk kelelahan,
stres, kesendirian, pencahayaan yang tidak memadai, dan kurangnya
pengalaman. Peningkatan penggunaan robot juga meningkatkan
produktivitas tenaga kerja sebesar 0,36\% per tahun, meningkatkan
produktivitas total, dan menurunkan harga \textit{output} \citep{5}. \citet{6}
mengungkap, antara tahun 2005 dan 2011, jumlah robot per 1.000
pekerja meningkat sebesar 25\%, yang menyebabkan penurunan 72.658
cedera kerja per tahun. Penurunan cedera ini diperkirakan menghemat
sekitar Rp27,3 triliun per tahun, atau total sekitar Rp191,2
triliun selama periode tersebut. Oleh
karena itu, penerapan metode otomatis yang tepat untuk mendeteksi
cacat dan kontaminan sangat krusial guna menjamin keamanan, mencapai
skalabilitas, dan meningkatkan efisiensi biaya.

Mayoritas model deteksi cacat menggunakan metode \textit{supervised
learning} seperti \textit{You Only Look Once} (YOLO). \citet{29}
mengusulkan model berbasis YOLOv8 yang dilengkapi modul
\textit{Multi-Path Convolution Attention} (MPCA) dan \textit{Partial
Self-Attention} (PSA) untuk meningkatkan akurasi dan sensitivitas
dalam mendeteksi cacat permukaan baja dan model ini diuji pada
\textit{dataset} publik NEU-DET dan VOC2007. \citet{30} mengembangkan
model YOLO untuk mendeteksi cacat pada \textit{magnet tile motor}
kendaraan listrik secara cepat dan akurat. \citet{31} juga
mengusulkan model YOLO yang dioptimalkan untuk mendeteksi cacat kecil
pada pengecoran logam dengan memanfaatkan modul deteksi objek kecil,
fitur re-ekstraksi, dan \textit{multi-scale attention}, serta diuji
pada \textit{dataset} NEU-DET. Semua pendekatan
tersebut menggunakan \textit{dataset} berlabel, sehingga memerlukan
data cacat yang sudah ditandai secara manual sebagai syarat pelatihan.

Algoritma \textit{deep learning} berupa \textit{autoencoder} dapat
dimanfaatkan untuk mendeteksi anomali dan cacat permukaan tanpa
\textit{dataset} berlabel \citep{7}.
Model \textit{autoencoder}, khususnya \textit{convolutional
autoencoder} (CAE), dapat dilatih hanya dengan sampel bebas cacat
\citep{8}. CAE bekerja dengan meminimalkan kesalahan rekonstruksi
sehingga fitur-fitur representatif dapat teridentifikasi secara
optimal. Model dilatih pada data normal, sehingga setiap sampel yang
mengandung cacat pada tahap evaluasi akan menunjukkan perbedaan yang
signifikan \citep{9}. Pendekatan ini sangat berguna ketika
pengumpulan data cacat terbukti sulit atau mahal.

YOLO telah digunakan bersama lengan robot untuk berbagai aplikasi.
\citet{10} mengembangkan model kontrol
lengan robot berbasis YOLO yang lebih efisien untuk menggenggam
komponen logam secara akurat di lingkungan industri yang kompleks.
\citet{11} mengusulkan metode deteksi objek
parsial dan estimasi kedalaman menggunakan kombinasi YOLO dan CNN
untuk mengontrol lengan robot, yang terbukti efektif dalam eksperimen
pengambilan objek dengan robot yang memiliki 4 derajat kebebasan.
\citet{12} menerapkan YOLO dalam
metode genggam robot untuk deteksi dan
pemilahan sampah secara \textit{real-time}, dengan pendekatan pembatasan area
pasca deteksi objek. \citet{13} mengembangkan versi
peningkatan dari YOLOv8n dengan integrasi modul \textit{dilated
re-parameterization}, \textit{feature pyramid}, dan
\textit{Scylla-IoU loss} untuk
meningkatkan akurasi dan adaptabilitas lengan robot pemetik apel
dalam kondisi kebun yang kompleks.

Sementara itu, \textit{autoencoder} adalah model jaringan saraf dalam
yang dilatih untuk merekonstruksi data \textit{input}, dan digunakan
dalam deteksi cacat dengan membandingkan perbedaan signifikan antara
\textit{input} normal dan hasil rekonstruksi dari data yang
mengandung cacat. \citet{14} menerapkan \textit{autoencoder} dengan
fungsi \textit{loss} berbasis \textit{Complex Wavelet Structural
Similarity} (CW-SSIM)
untuk mendeteksi anomali pada citra industri, yang terbukti lebih
efisien dibandingkan jaringan saraf besar lainnya. \citet{15}
mengusulkan pendekatan \textit{autoencoder} untuk mendeteksi kerusakan
pada lengan robot industri berdasarkan sinyal suara internal, dengan
memanfaatkan citra spektrogram \textit{Short-Time Fourier Transform}
(STFT). \citet{16} memperkenalkan model
deteksi anomali dengan struktur \textit{encoder-decoder-encoder}
(EDE) dan pelatihan dua tahap, yang
menggabungkan rekonstruksi dan pendekatan konfrontasi generatif.
\citet{17} mengembangkan FuseDecode \textit{autoencoder}, yang
menerapkan pembelajaran bertahap mulai dari tanpa supervisi,
semi-supervisi, hingga supervisi campuran, dan menunjukkan keunggulan
dalam deteksi cacat pada data industri nyata serta \textit{dataset}
MVTec AD. \citet{18} membandingkan klasifikasi biner
dan \textit{autoencoder} dalam deteksi anomali citra untuk proses
perakitan rangka, khususnya pada kesalahan posisi dan rotasi komponen, dan
menyimpulkan bahwa \textit{autoencoder} lebih unggul dalam mendeteksi anomali
halus serta fleksibel terhadap data yang terbatas.

Kebutuhan untuk meningkatkan efisiensi dan keamanan dalam inspeksi
industri mendorong pemanfaatan AI dan robotika. Deteksi cacat secara
otomatis, terutama dalam penanganan material berbahaya, sangat
penting untuk memastikan kualitas produk dan keselamatan manusia.
\textit{Autoencoder} merupakan algoritma yang efektif karena kemampuannya
untuk belajar dari data normal sekaligus mengidentifikasi anomali.
Dengan demikian, penelitian ini bertujuan untuk mengeksplorasi dan
mengoptimalkan penerapan \textit{autoencoder} dalam mendeteksi cacat
permukaan, sehingga dapat berkontribusi pada kemajuan manufaktur yang
cerdas dan berkelanjutan.

\vspace{1em}

\Needspace{3\baselineskip}
\section{Tujuan Penelitian}
Tujuan yang ingin dicapai dalam penelitian ini yaitu:
\begin{enumerate}
  \item Merancang dan melatih model deteksi objek berbasis YOLO yang
    mampu mengenali kontainer kimia secara akurat dalam berbagai kondisi.
  \item Membangun model deteksi kecacatan menggunakan algoritma
    \textit{convolutional autoencoder} yang efektif dalam membedakan antara
    kontainer cacat dan tidak.
  \item Mengintegrasikan model deteksi objek dan deteksi kecacatan ke
    dalam sistem berbasis lengan robot untuk proses identifikasi dan
    penyortiran secara otomatis.
\end{enumerate}

\vspace{1em}

\section{Manfaat Penelitian}
Manfaat yang ingin dicapai dalam penelitian ini yaitu:
\begin{enumerate}
  \item Memberikan kontribusi terhadap pengembangan sistem deteksi
    objek yang cepat dan akurat di lingkungan industri, khususnya
    untuk aplikasi inspeksi visual kontainer kimia dalam kondisi
    nyata yang bervariasi.
  \item Menyediakan solusi yang efektif dalam mendeteksi kecacatan secara
    otomatis, sehingga dapat menggantikan metode inspeksi manual yang
    memakan waktu dan rentan terhadap kesalahan manusia (\textit{human error}).
  \item Mendorong otomatisasi penuh dalam proses identifikasi dan
    penyortiran kontainer kimia, sehingga meningkatkan efisiensi
    produksi, menurunkan biaya operasional, dan memperkecil risiko
    kesalahan klasifikasi dalam sistem manufaktur.
\end{enumerate}
