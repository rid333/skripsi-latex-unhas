\chapter{Pendahuluan}
\section{Latar Belakang}
    Pada era digital saat ini, memiliki kendaraan pribadi seperti mobil telah menjadi
    kebutuhan pokok bagi banyak orang. Fenomena ini didorong oleh peningkatan jumlah
    penduduk global yang terus tumbuh secara signifikan setiap tahunnya, yang berimbas
    pada pertumbuhan jumlah kendaraan, terutama di kota-kota besar, khususnya di
    Makassar (Veeramanickam, 2022; Appiah, 2020). Jumlah kendaraan di Makassar telah
    meningkat signifikan selama beberapa tahun terakhir. Berdasarkan data terbaru, jumlah
    kendaraan di Makassar diperkirakan telah melebihi 1,7 juta unit sejak tahun 2021,
    dengan tren pertumbuhan yang terus berlanjut setiap tahun. Sementara itu, penduduk
    Makassar diperkirakan sekitar 1,5 juta jiwa, artinya jumlah kendaraan hampir
    sebanding dengan populasi. Pada tahun 2023, Makassar menghadapi masalah
    kemacetan yang semakin parah akibat pertumbuhan jumlah kendaraan yang tidak
    sebanding dengan kapasitas jalan. Data menunjukkan bahwa populasi kendaraan di
    Makassar mencapai sekitar 2,1 juta unit. Kenaikan ini berpotensi menciptakan masalah
    seperti kemacetan dan polusi udara yang berpengaruh pada kualitas hidup dan
    kesehatan masyarakat setempat (Aditya, 2023). \par

    Seiring dengan pertumbuhan jumlah kendaraan yang pesat, kebutuhan akan
    tempat parkir yang memadai semakin mendesak. Sistem parkir tradisional yang selama
    ini digunakan tidak mampu mengimbangi tingginya jumlah kendaraan yang
    memerlukan tempat parkir. Sistem ini sering kali menimbulkan masalah seperti
    kemacetan dan ketidaknyamanan bagi para pengguna kendaraan yang harus mencari
    tempat parkir dalam waktu lama. Selain itu, ketidakefisienan sistem parkir tradisional
    dapat menyebabkan antrian panjang di pintu keluar parkir, terutama saat jam sibuk
    (Aditya, 2023; Fahim, 2021; Suthir, 2022). Oleh karena itu, inovasi dalam sistem parkir,
    terutama yang berbasis teknologi, sangat diperlukan untuk memenuhi tuntutan
    masyarakat modern (Veeramanickam, 2022).\par
