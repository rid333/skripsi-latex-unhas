\chapter{Metode Penelitian}
\section{Tempat dan Waktu Penelitian}
Penelitian ini dilaksanakan mulai dari bulan Februari 2025 hingga Juni
2025, bertempat di Laboratorium Elektronika dan Instrumentasi,
Departemen Fisika, Fakultas Matematika dan Ilmu Pengetahuan Alam,
Universitas Hasanuddin, Makassar. \\

\section{Peralatan Penelitian}
Adapun peralatan yang digunakan pada penelitian ini adalah sebagai berikut
\begin{enumerate}
  \item Arduino Uno berf ungsi sebagai mikrokontroler utama yang
    mengendalikan motor servo pada lengan robot serta menerima sinyal
    dari sensor.
  \item Motor Servo digunakan sebagai aktuator untuk menggerakkan
    bagian-bagian lengan robot sesuai perintah dari Arduino.
  \item Lengan Robot EEZYbotARM MK1 berfungsi sebagai struktur
    mekanik yang menjadi tempat pemasangan motor servo dan berperan
    sebagai sistem pergerakan robotik.
  \item Power Supply 5V berf ungsi memberikan catu daya stabil untuk
    motor servo agar dapat beroperasi dengan baik.
  \item Sensor PIR HC-SR501 mendeteksi gerakan dan membantu
    menghitung jumlah kontainer cacat dan non-cacat yang lewat.
  \item Kamera digunakan untuk mengambil gambar kontainer kimia, yang
    kemudian diproses oleh model deteksi (YOLO) dan deteksi cacat (Autoencoder).
  \item Laptop/Komputer digunakan untuk mengunggah program ke Arduino
    Uno, serta menjalankan model deteksi berbasis YOLO dan
    Autoencoder untuk analisis visual.
  \item Kabel Jumper berfungsi menghubungkan berbagai komponen
    elektronik seperti sensor dan aktuator ke papan rangkaian dan
    Arduino.Kabel Jumper berfungs imenghubungkan setiap komponen sensor.
  \item Papan rangkaian berfungsi untuk menyediakan jalur koneksi
    antar komponen. \\
\end{enumerate}

\section{Metode Kerja}
Dalam penelitian ini terdapat beberapa tahapan yang harus dilakukan,
tahapan penelitian dapat dilihat pada flowchart Gambar 1. Penelitian
ini dibatasi pada perancangan dan pembuatan prototipe sistem deteksi cacat.
