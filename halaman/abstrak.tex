\begingroup
\singlespacing
\chapter*{ABSTRAK}
\addcontentsline{toc}{chapter}{ABSTRAK}

\noindent
Alridho. \textbf{\JudulSkripsi} (dibimbing oleh Prof. Dr. Arifin, M.T.). \par

\vspace*{0.1cm}

\noindent
\textbf{Latar Belakang}. Inspeksi manual kontainer kimia memiliki tingkat kesalahan tinggi (20\%–30\%) dan berisiko bagi pekerja. \textit{Artificial Intelligence} (AI) dan robotika dapat meningkatkan efisiensi dan keselamatan. Model deteksi cacat berbasis \textit{autoencoder} efektif karena bisa belajar dari data normal untuk mendeteksi anomali, sangat membantu saat data cacat terbatas. \textbf{Tujuan}. Penelitian ini bertujuan: (1) merancang model deteksi objek berbasis YOLO untuk mengenali kontainer, (2) membangun model deteksi cacat menggunakan CVAE, dan (3) mengintegrasikan keduanya dalam sistem lengan robot untuk identifikasi dan penyortiran otomatis. \textbf{Metode}. Penelitian menggunakan pendekatan eksperimental dengan prototipe yang terdiri dari lengan robot, Arduino Uno, motor \textit{servo}, sensor PIR, dan kamera. Kamera menangkap citra kontainer yang kemudian dideteksi YOLO. CVAE menganalisis cacat berdasarkan \textit{reconstruction error}, dengan ambang 0{,}007183 (dari kurva ROC). Lengan robot lalu menyortir kontainer ke area "cacat" atau "normal". \textbf{Hasil}. Model YOLO mencapai mAP@50 dan mAP@50-95 mendekati 1.00. Model CVAE berhasil membedakan kontainer cacat dan normal. Pengujian 25 kali menunjukkan akurasi 100\%, dan seluruh proses ditampilkan \textit{real-time} berbasis web. \textbf{Kesimpulan}. Prototipe ini berhasil mendemonstrasikan otomasi penuh inspeksi visual di lingkungan industri, dengan akurasi tinggi dan sistem penyortiran robotik yang terintegrasi. \par

\vspace*{0.1cm}

\noindent
\textbf{Kata Kunci:} \textit{Deteksi cacat; prototipe; lengan robot; YOLO; autoencoder; IoT.}

\endgroup

\newpage

\begingroup
\singlespacing
\chapter*{ABSTRACT}
\addcontentsline{toc}{chapter}{ABSTRACT}

\noindent
Alridho. \textbf{PROTOTYPE OF DEEP LEARNING-BASED DEFECT DETECTION AND SORTING SYSTEM USING ROBOTIC ARM FOR CHEMICAL CONTAINERS IN ISOLATED INDUSTRIAL AREAS} (dibimbing oleh Prof. Dr. Arifin, M.T.). \par

\vspace*{0.1cm}
 
\noindent
\textbf{Background}. Manual inspection of chemical containers has a high error rate (20\%–30\%) and poses risks to workers. Artificial Intelligence (AI) and robotics can improve efficiency and safety. Defect detection models based on autoencoders are effective since they can learn from normal data to detect anomalies, which is very useful when defective data are limited. \textbf{Objective}. This study aims to: (1) design an object detection model based on YOLO to accurately recognize containers, (2) develop a defect detection model using CVAE, and (3) integrate both models into a robotic arm system for automatic identification and sorting. \textbf{Method}. The research uses an experimental approach with a prototype consisting of a robotic arm, Arduino Uno, servo motors, PIR sensor, and camera. The camera captures images of the containers, which are then detected by YOLO. The CVAE analyzes defects based on reconstruction error, with a threshold of 0.007183 (from the ROC curve). The robotic arm then sorts the containers into "defect" or "normal" areas. \textbf{Results}. The YOLO model achieved mAP@50 and mAP@50-95 close to 1.00. The CVAE model successfully distinguished defective and normal containers. Testing conducted 25 times showed 100\% accuracy, and the entire process was displayed in real-time via a web-based interface. \textbf{Conclusion}. This prototype successfully demonstrated full automation of visual inspection in an industrial environment, with high accuracy and an integrated robotic sorting system.


\vspace*{0.1cm}

\noindent
\textbf{Keywords:} \textit{Defect detection; prototype; robotic arm; YOLO; autoencoder; IoT}.

\endgroup
