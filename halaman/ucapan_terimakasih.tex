\begingroup
\chapter*{UCAPAN TERIMA KASIH}
\addcontentsline{toc}{chapter}{UCAPAN TERIMA KASIH}
% \lipsum[1] \par
% \lipsum[2] \par
% Lorem ipsum dolor sit amet, consectetur adipiscing elit. Fusce
% malesuada non dui quis condimentum. Etiam dapibus ligula sapien:
% \begin{enumerate}
%   \item Items are numbered automatically.
%   \item The numbers start at 1 with each use of the
%     \texttt{enumerate} environment.
%   \item Another entry in the list.\\
% \end{enumerate}

\noindent
Segala puji dan syukur penulis panjatkan kehadirat Allah Subhanahu Wa
Ta’ala atas segala berkah, rahmat, dan karunia-Nya yang telah
memberikan ilmu pengetahuan, pengalaman, kekuatan, kesabaran, dan
kesempatan kepada penulis sehingga mampu menyelesaikan skripsi ini.
Penulisan skripsi yang berjudul \textbf{"\JudulSkripsi"} merupakan
upaya penulis memenuhi salah satu syarat dalam menyelesaikan pendidikan dan
memperoleh gelar Sarjana Sains di Departemen Fisika, Fakultas
Matematika dan Ilmu Pengetahuan Alam, Universitas Hasanuddin. Selain
itu, skripsi ini juga diharapkan dapat memberikan manfaat bagi
pembaca dan peneliti lain untuk menambah wawasan dalam bidang fisika
khususnya elektronika dan instrumentasi.

Proses penyelesaian skripsi ini merupakan suatu rangkaian perjuangan
yang cukup panjang bagi penulis. Selama proses penelitian maupun
penyusunan skripsi ini, tidak sedikit hambatan maupun kendala yang
penulis hadapi. Do’a dan dukungan dari berbagai pihak merupakan hal
yang berarti, sehingga penyusunan skripsi ini dapat diselesaikan oleh
penulis. Oleh karena itu, dengan tulus dan ikhlas, penulis
mengucapkan terima kasih sebanyak-banyaknya.

Penulis menyampaikan penghargaan setinggi-tingginya dan banyak terima
kasih kepada Bapak \textbf{Prof. Dr. Arifin, M.T} selaku pembimbing saya atas
kesediannya telah meluangkan banyak waktu, tenaga dan pikiran dalam
memberikan bimbingan dan motivasi kepada Penulis, mulai dari awal
penyusunan sampai penyelesaian skripsi ini. Pada kesempatan ini pula,
dengan segala kerendahan hati Penulis ingin menyampaikan ucapan
terima kasih yang sebesar-besarnya kepada:

\begin{enumerate}
  \item Keluarga tercinta, khususnya kedua orang tua saya, Ayahanda
    \textbf{Muh. Anto} dan Ibunda \textbf{Nuraeni}, yang telah
    memberikan dukungan materiil dan moril serta tak henti mendoakan.
    Terima kasih juga untuk adik saya, Cinta Kasih Rahmadany, yang
    turut memberi semangat dalam menyelesaikan perjalanan ini.
  \item Dosen penguji kak \textbf{Ida Laila, S.Si., M.Si.} dan Ibu
    \textbf{Prof. Dr. Sri Suryani, DEA.} atas kesediaannya untuk
    meluangkan waktu, menguji, serta memberikan kritik dan saran yang
    sangat berharga untuk skripsi ini.
  \item Bapak dan Ibu \textbf{Dosen Pengajar Departemen Fisika Fakultas
    Matematika dan Ilmu Pengetahuan Alam Universitas Hasanuddin},
    terima kasih atas ilmu dan bimbingannya selama ini. Semoga hasil
    ajaran Bapak dan Ibu selalu memberikan manfaat bagi setiap orang.
  \item Bapak dan Ibu Staf Departemen Fisika khususnya Pak \textbf{Syukur},
    Ibu \textbf{Evi}, dan Kak \textbf{Rana} yang selalu membantu penulis
    selama berada di kampus.
  \item Teman-teman seperjuangan di kampus \textbf{Ragil, Vivaldo, Salim,
    Werdi, Adri, Akmal, dan lain-lain} atas kebersamaannya di dalam
    dan di luar kampus.
  \item Seluruh pihak yang telah membantu dan memberikan dukungan
    yang tidak dapat penulis sebutkan namanya satu per satu.
\end{enumerate}

Penulis berharap semoga segala kebaikan yang diberikan dari berbagai
pihak kepada penulis untuk menyelesaikan skripsi ini dapat bernilai
ibadah. Akhir kata, penulis memohon maaf atas kesalahan yang
disengaja maupun tidak disengaja dalam rangkaian penyusunan skripsi
ini. Semoga skripsi ini dapat memberikan manfaat bagi pembaca.

\vspace{2cm}

\hfill
\begin{minipage}{0.4\textwidth}
  \raggedleft
  Makassar, \today \par
  \vspace{2cm}
  Alridho \\
  H021211006
\end{minipage}
\endgroup
