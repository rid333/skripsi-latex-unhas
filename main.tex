\documentclass{skripsi}

\newcommand{\JudulSkripsi}{PROTOTIPE SISTEM DETEKSI DAN PENANGANAN
  CACAT PADA KONTAINER KIMIA UNTUK AREA INDUSTRI-TERISOLASI BERBASIS
ARTIFICIAL INTELLIGENCE (AI) DAN LENGAN ROBOT}
\newcommand{\NamaPenulis}{ALRIDHO}
\newcommand{\NIM}{H021211006}

% Flowchart boxes
\usetikzlibrary{shapes.geometric, arrows, positioning, calc}

% Flowchart
\tikzstyle{startstop} = [rectangle, rounded corners=10pt, minimum
  width=3cm, minimum height=1cm,text centered, draw=black,
, line width=1pt]
\tikzstyle{process} = [rectangle, minimum
  width=3cm, minimum height=1cm,text centered, draw=black,
, line width=1pt]
\tikzstyle{decision} = [diamond, minimum
  width=3cm, minimum height=1cm,text centered, draw=black,
, line width=1pt]
\tikzstyle{input} = [trapezium, trapezium left angle=70, trapezium
  right angle=110,
  minimum width=3cm, minimum height=1cm, text centered, draw=black,
line width=1pt]

% Pipeline
\tikzstyle{data} = [
  cylinder,
  shape border rotate=0,
  aspect=0.25,
  minimum height=2cm,
  minimum width=1cm,
  text centered,
  draw=black,
  fill=cyan!50,
  line width=1pt
]
\tikzstyle{pipeline-box} = [
  rectangle,
  rounded corners=5pt,
  minimum width=2cm,
  minimum height=1cm,
  text centered,
  draw=black,
  line width=1pt
]

\tikzstyle{arrow} = [rounded corners=5pt, thick,->,>=stealth]

\begin{document}
\newgeometry{top=22.5mm, bottom=22.5mm}
\definecolor{sampul}{HTML}{c80404}
\begingroup
\singlespacing
\fontsize{11pt}{13pt}\selectfont
\begin{titlepage}
  \begin{tikzpicture}[remember picture, overlay]
    \fill[sampul] (current page.south west) rectangle (current page.north east);
  \end{tikzpicture}

  \color{black}

  \vspace*{-0.7cm}

  \begin{center}
    \textbf{\JudulSkripsi}
  \end{center}

  \vfill
  % Hexagon
  \begin{center}
    \begin{tikzpicture}[scale=1]
      \foreach \i/\x/\y in {
        1/3/3.2,
        2/7.8/3.2,
        3/0.6/4.6,
        4/5.4/4.6,
        5/3/6,
        6/5.4/7.4
      } {
        \begin{scope}[shift={(\x,\y)}]
          \clip (0:1.5) \foreach \a in {60,120,...,360} { -- (\a:1.5) };
          \node at (0,0) {\includegraphics[width=4cm]{gambar/img\i.jpg}};
          \draw[white, line width=1.5pt] (0:1.5) \foreach \a in
          {60,120,...,360} { -- (\a:1.5) } -- cycle;
        \end{scope}
      }
    \end{tikzpicture}
  \end{center}

  \vfill

  \begin{center}
    \textbf{\NamaPenulis} \\
    \textbf{\NIM} \\[1cm]
  \end{center}

  \vfill

  \begin{minipage}{0.15\textwidth}
    \hspace*{-1cm}
    \includegraphics[width=2.5cm]{gambar/uh-b.png}
  \end{minipage}
  \hfill
  \begin{minipage}{0.8\textwidth}
    \centering
    \textbf{PROGRAM STUDI FISIKA} \\
    \textbf{FAKULTAS MATEMATIKA DAN ILMU PENGETAHUAN ALAM} \\
    \textbf{UNIVERSITAS HASANUDDIN} \\
    \textbf{MAKASSAR} \\
    \textbf{2025}
  \end{minipage}

  \vspace*{0.5cm}

\end{titlepage}
\endgroup
\restoregeometry

\newgeometry{top=22.5mm, bottom=22.5mm}
\begingroup
\fontsize{11pt}{13pt}\selectfont
\begin{center}
    \textbf{JUDUL BAHASA INDONESIA} \\
    \vfill
    \textbf{ALRIDHO} \\
    \textbf{H021211006} \\
    \vspace*{1.5cm}
    \includegraphics[height=3.2cm]{gambar/uh-fc.png} \\
    \vfill
    \textbf{PROGRAM STUDI FISIKA} \\
    \textbf{FAKULTAS MATEMATIKA DAN ILMU PENGETAHUAN ALAM} \\
    \textbf{UNIVERSITAS HASANUDDIN} \\
    \textbf{MAKASSAR} \\
    \textbf{2025}
\end{center}
\endgroup
\restoregeometry

\newgeometry{top=22.5mm, bottom=22.5mm}
\addcontentsline{toc}{chapter}{HALAMAN PENGAJUAN}
\begingroup
\singlespacing
\fontsize{11pt}{13pt}\selectfont
\begin{center}
  \textbf{\JudulSkripsi} \\
  \vfill
  \NamaPenulis \\
  \NIM \\
  \vfill
  \begingroup
  \fontsize{10pt}{13pt}\selectfont
  Skripsi \\
  \vspace*{1cm}
  sebagai salah satu syarat untuk mencapai gelar sarjana \\
  \endgroup
  \vspace*{1cm}
  Program Studi Fisika \\
  \vfill
  pada \\
  \vfill
  \textbf{PROGRAM STUDI FISIKA} \\
  \textbf{FAKULTAS MATEMATIKA DAN ILMU PENGETAHUAN ALAM} \\
  \textbf{UNIVERSITAS HASANUDDIN} \\
  \textbf{MAKASSAR} \\
  \textbf{2025}
\end{center}
\endgroup
\restoregeometry

\backgroundsetup{
  scale=1,
  angle=0,
  opacity=0.2,
  position=current page.center,
  contents={\includegraphics[width=7cm,height=9cm]{gambar/uh-fc.png}}
}

\newgeometry{top=22.5mm, bottom=22.5mm}
\addcontentsline{toc}{chapter}{HALAMAN PENGESAHAN}
\begingroup
\singlespacing
\fontsize{11pt}{13pt}\selectfont
\begin{center}
  \textbf{SKRIPSI} \\
  \vfill
  \textbf{\JudulSkripsi} \\
  \vfill
  \textbf{\underline{\NamaPenulis}} \\
  \textbf{\NIM} \\
  \vfill
  \begingroup
  \fontsize{10pt}{13pt}\selectfont
  Skripsi, \\
  \vfill
  telah dipertahankan di depan Panitia Ujian Sarjana Xxxx pada
  tanggal bulan tahun dan dinyatakan telah memenuhi syarat kelulusan \\
  pada \\
  \vfill
  Program Studi Fisika \\
  Departemen Fisika \\
  Fakultas Matematika dan Ilmu Pengetahuan Alam \\
  Universitas Hasanuddin \\
  Makassar \\
  \vfill
  \noindent
  \begin{minipage}[t]{0.38\textwidth}
    Mengesahkan:\par
    Pembimbing tugas akhir,\par\vspace{2cm}

    \underline{Prof. Dr. Arifin, M.T}\\
    NIP. 19670520 199403 1 002
  \end{minipage}
  \hfill
  \begin{minipage}[t]{0.38\textwidth}
    Mengetahui:\par
    Ketua Program Studi,\par\vspace{2cm}

    \underline{Prof. Dr. Arifin, M.T}\\
    NIP. 19670520 199403 1 002
  \end{minipage}
  \endgroup
\end{center}

\endgroup
\restoregeometry
\clearpage
\backgroundsetup{contents={}}

\begingroup
\singlespacing
\chapter*{PERNYATAAN KEASLIAN SKRIPSI}
\addcontentsline{toc}{chapter}{LEMBAR PERNYATAAN KEASLIAN SKRIPSI}
\noindent
Dengan ini saya menyatakan bahwa, skripsi berjudul “\textbf{\JudulSkripsi}”
adalah benar karya saya dengan arahan dari pembimbing (Prof. Dr.
Arifin, M.T). Karya ilmiah ini belum diajukan dan tidak sedang
diajukan dalam bentuk apapun kepada perguruan tinggi mana pun. Sumber
informasi yang berasal atau dikutip dari karya yang diterbitkan
maupun tidak diterbitkan dari penulis lain telah disebutkan dalam
teks dan dicantumkan dalam Daftar Pustaka skripsi ini. Apabila di
kemudian hari terbukti atau dapat dibuktikan bahwa sebagian atau
keselurusan skripsi ini adalah karya orang lain, maka saya bersedia
menerima sanksi atas perbuatan tersebut berdasarkan aturan yang berlaku. \par

\noindent
\\
Dengan ini saya melimpahkan hak cipta (hak ekonomis) dari karya tulis
saya berupa skripsi ini kepada Universitas Hasanuddin. \par

\vspace{2cm}

\hfill
\begin{minipage}{0.4\textwidth}
  \raggedleft
  Makassar, 14 Juli 2025 \par
  \vspace{2cm}
  Alridho \\
  H021211006
\end{minipage}
\endgroup

\begingroup
\chapter*{UCAPAN TERIMA KASIH}
\addcontentsline{toc}{chapter}{UCAPAN TERIMA KASIH}
% \lipsum[1] \par
% \lipsum[2] \par
% Lorem ipsum dolor sit amet, consectetur adipiscing elit. Fusce
% malesuada non dui quis condimentum. Etiam dapibus ligula sapien:
% \begin{enumerate}
%   \item Items are numbered automatically.
%   \item The numbers start at 1 with each use of the
%     \texttt{enumerate} environment.
%   \item Another entry in the list.\\
% \end{enumerate}

\noindent
Segala puji dan syukur penulis panjatkan kehadirat Allah Subhanahu Wa
Ta’ala atas segala berkah, rahmat, dan karunia-Nya yang telah
memberikan ilmu pengetahuan, pengalaman, kekuatan, kesabaran, dan
kesempatan kepada penulis sehingga mampu menyelesaikan skripsi ini.
Penulisan skripsi yang berjudul \textbf{"\JudulSkripsi"} merupakan
upaya penulis memenuhi salah satu syarat dalam menyelesaikan pendidikan dan
memperoleh gelar Sarjana Sains di Departemen Fisika, Fakultas
Matematika dan Ilmu Pengetahuan Alam, Universitas Hasanuddin. Selain
itu, skripsi ini juga diharapkan dapat memberikan manfaat bagi
pembaca dan peneliti lain untuk menambah wawasan dalam bidang fisika
khususnya elektronika dan instrumentasi.

Proses penyelesaian skripsi ini merupakan suatu rangkaian perjuangan
yang cukup panjang bagi penulis. Selama proses penelitian maupun
penyusunan skripsi ini, tidak sedikit hambatan maupun kendala yang
penulis hadapi. Do’a dan dukungan dari berbagai pihak merupakan hal
yang berarti, sehingga penyusunan skripsi ini dapat diselesaikan oleh
penulis. Oleh karena itu, dengan tulus dan ikhlas, penulis
mengucapkan terima kasih sebanyak-banyaknya.

Penulis menyampaikan penghargaan setinggi-tingginya dan banyak terima
kasih kepada Bapak \textbf{Prof. Dr. Arifin, M.T} selaku pembimbing saya atas
kesediannya telah meluangkan banyak waktu, tenaga dan pikiran dalam
memberikan bimbingan dan motivasi kepada Penulis, mulai dari awal
penyusunan sampai penyelesaian skripsi ini. Pada kesempatan ini pula,
dengan segala kerendahan hati Penulis ingin menyampaikan ucapan
terima kasih yang sebesar-besarnya kepada:

\begin{enumerate}
  \item Keluarga tercinta, khususnya kedua orang tua saya, Ayahanda
    \textbf{Muh. Anto} dan Ibunda \textbf{Nuraeni}, yang telah
    memberikan dukungan materiil dan moril serta tak henti mendoakan.
    Terima kasih juga untuk adik saya, Cinta Kasih Rahmadany, yang
    turut memberi semangat dalam menyelesaikan perjalanan ini.
  \item Dosen penguji kak \textbf{Ida Laila, S.Si., M.Si.} dan Ibu
    \textbf{Prof. Dr. Sri Suryani, DEA.} atas kesediaannya untuk
    meluangkan waktu, menguji, serta memberikan kritik dan saran yang
    sangat berharga untuk skripsi ini.
  \item Bapak dan Ibu \textbf{Dosen Pengajar Departemen Fisika Fakultas
    Matematika dan Ilmu Pengetahuan Alam Universitas Hasanuddin},
    terima kasih atas ilmu dan bimbingannya selama ini. Semoga hasil
    ajaran Bapak dan Ibu selalu memberikan manfaat bagi setiap orang.
  \item Bapak dan Ibu Staf Departemen Fisika khususnya Pak \textbf{Syukur},
    Ibu \textbf{Evi}, dan Kak \textbf{Rana} yang selalu membantu penulis
    selama berada di kampus.
  \item Teman-teman seperjuangan di kampus \textbf{Ragil, Vivaldo, Salim,
    Werdi, Adri, Akmal, dan lain-lain} atas kebersamaannya di dalam
    dan di luar kampus.
  \item Seluruh pihak yang telah membantu dan memberikan dukungan
    yang tidak dapat penulis sebutkan namanya satu per satu.
\end{enumerate}

Penulis berharap semoga segala kebaikan yang diberikan dari berbagai
pihak kepada penulis untuk menyelesaikan skripsi ini dapat bernilai
ibadah. Akhir kata, penulis memohon maaf atas kesalahan yang
disengaja maupun tidak disengaja dalam rangkaian penyusunan skripsi
ini. Semoga skripsi ini dapat memberikan manfaat bagi pembaca.

\vspace{2cm}

\hfill
\begin{minipage}{0.4\textwidth}
  \raggedleft
  Makassar, \today \par
  \vspace{2cm}
  Alridho \\
  H021211006
\end{minipage}
\endgroup

\begingroup
\singlespacing
\chapter*{ABSTRAK}
\addcontentsline{toc}{chapter}{ABSTRAK}

\noindent
Alridho. \textbf{JUDUL BAHASA INDONESIA} (dibimbing oleh Prof. Dr. Arifin, M.T.). \par

\vspace*{0.1cm}

\noindent
\lipsum[1] \par

\vspace*{0.1cm}

\noindent
\textbf{Kata Kunci:} Prototipe; monitoring; IoT; sensor; telegram.

\endgroup

\newpage

\begingroup
\singlespacing
\chapter*{ABSTRACT}
\addcontentsline{toc}{chapter}{ABSTRACT}

\noindent
Alridho. \textbf{JUDUL BAHASA INGRRIS} (dibimbing oleh Prof. Dr. Arifin, M.T.). \par

\vspace*{0.1cm}
 
\noindent
\lipsum[1]


\vspace*{0.1cm}

\noindent
\textbf{Kata Kunci:} Prototipe; monitoring; IoT; sensor; telegram.

\endgroup


\newpage
\chapter*{DAFTAR ISI}
\noindent\hfill\textbf{Halaman}
\addcontentsline{toc}{chapter}{DAFTAR ISI}
\tableofcontents

\newpage
\chapter*{DAFTAR TABEL}
\noindent\hfill\textbf{Halaman}
\addcontentsline{toc}{chapter}{DAFTAR TABEL}
\listoftables

\newpage
\chapter*{DAFTAR GAMBAR}
\noindent\hfill\textbf{Halaman}
\addcontentsline{toc}{chapter}{DAFTAR GAMBAR}
\listoffigures

\newpage
\chapter*{DAFTAR LAMPIRAN}
\noindent\hfill\textbf{Halaman}
\addcontentsline{toc}{chapter}{DAFTAR LAMPIRAN}
\begin{spacing}{1}
  \begin{itemize}
      \setlength{\itemsep}{1pt}
      \renewcommand\labelitemi{}
    \item \makebox[12cm][l]{\textbf{Lampiran 1.} Peralatan penelitian \dotfill}
      \pageref{Lampiran 1}
    \item \makebox[12cm][l]{\textbf{Lampiran 2.} Program untuk latih
      model YOLO \dotfill}
      \pageref{Lampiran 2}
    \item \makebox[12cm][l]{\textbf{Lampiran 3.} Program untuk latih
      model autoencoder \dotfill}
      \pageref{Lampiran 3}
    \item \makebox[12cm][l]{\textbf{Lampiran 4.} Program kombinasi YOLO
      dan autoencoder \dotfill}
      \pageref{Lampiran 4}
    \item \makebox[12cm][l]{\textbf{Lampiran 5.} Program lengan robot \dotfill}
      \pageref{Lampiran 5}
    \item \makebox[12cm][l]{\textbf{Lampiran 6.} Dokumentasi pergerakan
      lengan robot \dotfill}
      \pageref{Lampiran 6}
  \end{itemize}
\end{spacing}

\chapter{PENDAHULUAN}
\pagenumbering{arabic}
\section{Latar Belakang}
\noindent
Seiring dengan pesatnya perkembangan AI,
robotika, dan otomatisasi, penggunaan lengan robot otomatis semakin
meluas di sektor manufaktur, logistik, dan layanan \citep{1}. Otomatisasi
bertujuan untuk menjalankan rangkaian tindakan sesuai dengan proses
yang telah ditetapkan tanpa intervensi manusia dengan mengendalikan
perangkat mekanis secara otomatis \citep{2}. Lengan robot mampu
melakukan berbagai tugas seperti perakitan, penanganan, dan
pengemasan sehingga dapat menggantikan pekerjaan yang bersifat
berulang dan melelahkan bagi manusia \citep{3}. Dalam industri
manufaktur, lengan robot sering
dikombinasikan dengan sensor kamera, algoritma visi komputer, dan
teknologi otomatisasi guna mendeteksi cacat pada objek dengan tingkat
presisi dan efisiensi yang tinggi.

Dalam konteks industri, terutama pada penanganan material berbahaya
seperti kontainer kimia, aspek keselamatan menjadi prioritas utama.
Inspeksi visual otomatis dengan bantuan robot diperlukan untuk
mengurangi risiko paparan zat berbahaya terhadap manusia. Isolasi
dalam proses deteksi cacat pada kontainer kimia sangat
penting, karena kerusakan atau kontaminasi pada kontainer dapat
menimbulkan ancaman keselamatan bagi konsumen. \citet{4} melaporkan
bahwa kesalahan
inspeksi manual berkisar 20\% hingga 30\% yang disebabkan oleh
beberapa faktor seperti kelelahan,
stres, pencahayaan yang tidak memadai, dan kurangnya
pengalaman. Peningkatan penggunaan robot dalam industri juga
berdampak positif terhadap produktivitas. \citet{5} menemukan bahwa
otomatisasi meningkatkan produktivitas tenaga kerja sebesar 0,36\%
per tahun, serta menurunkan harga produksi. \citet{6}
melaporkan bahwa antara 2005 dan 2011, jumlah robot per 1.000
pekerja meningkat sebesar 25\%, yang berkontribusi terhadap penurunan
72.658 kasus cedera kerja per tahun dengan potensi penghematan biaya
mencapai sekitar Rp27,3 triliun per tahun, atau total sekitar Rp191,2
triliun selama periode tersebut. Oleh
karena itu, penerapan metode otomatis yang tepat dalam mendeteksi
cacat dan kontaminasi sangat krusial untuk menjamin keselamatan
kerja, mencapai skalabilitas sistem, dan meningkatkan efisiensi biaya.

Mayoritas sistem model deteksi cacat saat ini menggunakan metode
\textit{supervised learning} seperti algoritma \textit{You Only Look
Once} (YOLO). \citet{29} mengusulkan model berbasis YOLOv8 dengan
integrasi modul \textit{Multi-Path Convolution Attention} (MPCA) dan
\textit{Partial Self-Attention} (PSA) untuk meningkatkan akurasi dan
sensitivitas dalam mendeteksi cacat permukaan baja yang telah diuji pada
\textit{dataset} publik \textit{Northeastern University-Defect Detection}
(NEU-DET) dan VOC2007. \citet{30} mengembangkan model YOLO untuk
mendeteksi cacat
pada komponen \textit{magnet tile motor} kendaraan listrik dengan cepat dan
akurat. \citet{31} juga mengusulkan model YOLO yang dioptimalkan
untuk mendeteksi cacat kecil pada pengecoran logam dengan
menggunakan modul deteksi objek kecil, fitur re-ekstraksi, dan
\textit{multi-scale attention}, serta diuji menggunakan \textit{dataset}
NEU-DET. Semua pendekatan tersebut menggunakan \textit{dataset}
berlabel, sehingga memerlukan data cacat yang sudah ditandai secara
manual sebagai prasyarat pelatihan.

Sebagai alternatif, algoritma \textit{deep learning} berupa
\textit{autoencoder} dapat dimanfaatkan untuk mendeteksi anomali dan
cacat permukaan tanpa memerlukan \textit{dataset} berlabel \citep{7}.
Model \textit{autoencoder}, khususnya \textit{convolutional
autoencoder} (CAE), dapat dilatih hanya dengan sampel data bebas cacat
\citep{8}. CAE bekerja dengan meminimalkan kesalahan rekonstruksi
sehingga fitur-fitur representatif dari objek dapat teridentifikasi
secara optimal. Karena model hanya dilatih menggunakan data normal,
setiap sampel yang mengandung cacat akan menunjukkan deviasi
signifikan saat proses evaluasi \citep{9}. Pendekatan ini sangat
berguna ketika pengumpulan data sulit atau mahal untuk dikumpulkan.

YOLO juga telah banyak digunakan bersama lengan robot dalam berbagai
aplikasi. \citet{10} mengembangkan sistem kontrol
lengan robot berbasis YOLO yang efisien untuk menggenggam
komponen logam secara presisi di lingkungan industri yang kompleks.
\citet{11} mengusulkan metode yang menggabungkan YOLO dan
\textit{Convolutional Neural Network} (CNN) untuk mendeteksi objek
parsial dan memperkirakan kedalaman serta mengontrol lengan robot
dengan 4 derajat kebebasan. \citet{12} menerapkan YOLO dalam metode
genggam robot untuk deteksi dan pemilahan sampah secara
\textit{real-time}, dengan pembatasan area pasca deteksi
objek. \citet{13} mengembangkan YOLOv8n yang
telah ditingkatkan dengan integrasi modul \textit{dilated
re-parameterization}, \textit{feature pyramid}, dan
\textit{Scylla-IoU loss} yang terbukti meningkatkan akurasi dan
adaptabilitas lengan robot dalam memetik apel di kebun yang kompleks.

Sementara itu, \textit{autoencoder} sebagai model jaringan saraf
dilatih untuk merekonstruksi data \textit{input} dan digunakan
dalam deteksi cacat dengan membandingkan perbedaan signifikan antara
\textit{input} normal dan hasil rekonstruksi.
\citet{14} menggunakan \textit{autoencoder} dengan
fungsi \textit{loss} berbasis \textit{Complex Wavelet Structural
Similarity} (CW-SSIM) untuk mendeteksi anomali citra industri, yang
terbukti efisien dibandingkan model \textit{deep learning} lainnya.
\citet{15} mengembangkan \textit{autoencoder} untuk mendeteksi kerusakan
pada lengan robot berdasarkan sinyal suara internal, menggunakan
citra spektrogram \textit{Short-Time Fourier Transform}
(STFT). \citet{16} mengusulkan model
deteksi anomali dengan arsitektur \textit{encoder-decoder-encoder}
(EDE) dan pelatihan dua tahap, yang
menggabungkan pendekatan rekonstruksi dan konfrontasi generatif.
\citet{17} mengembangkan \textit{FuseDecode} \textit{autoencoder},
dengan strategi pembelajaran bertahap mulai dari tanpa supervisi,
semi-supervisi, dan supervisi campuran, serta menunjukkan kinerja unggul
dalam deteksi cacat pada data industri nyata dan \textit{dataset}
MVTec \textit{Anomaly Detection} (AD). \citet{18} membandingkan
klasifikasi biner dan \textit{autoencoder} dalam deteksi anomali
citra untuk proses perakitan rangka, khususnya pada kesalahan posisi
dan rotasi komponen, dan menyimpulkan bahwa \textit{autoencoder}
lebih efektif untuk deteksi anomali halus dan fleksibel dalam kondisi
data yang terbatas.

Kebutuhan untuk meningkatkan efisiensi dan keselamatan dalam proses inspeksi
industri mendorong pemanfaatan AI dan robotika. Deteksi cacat
otomatis, terutama dalam penanganan material berbahaya, sangat
penting untuk menjamin kualitas produk dan keselamatan manusia.
\textit{Autoencoder} merupakan algoritma pembelajaran mesin yang
efektif karena kemampuannya untuk belajar dari data normal dan
mengidentifikasi anomalis secara mandiri.
Dengan demikian, penelitian ini bertujuan untuk mengeksplorasi dan
mengoptimalkan penerapan \textit{autoencoder} dalam mendeteksi cacat
permukaan, sehingga dapat mendukung pengembangan manufaktur
cerdas yang efisien dan berkelanjutan.

\vspace{1em}

\Needspace{3\baselineskip}
\section{Tujuan Penelitian}
\noindent
Penelitian ini bertujuan untuk:
\begin{enumerate}
  \item Merancang dan melatih model deteksi objek berbasis YOLOv12 yang
    mampu mengenali kontainer kimia secara akurat dalam berbagai lingkungan.
  \item Membangun model deteksi kecacatan menggunakan algoritma
    \textit{convolutional autoencoder} yang efektif dalam membedakan
    antara kontainer cacat dan tidak.
  \item Mengintegrasikan model deteksi objek dan deteksi kecacatan ke
    dalam sistem berbasis lengan robot untuk proses identifikasi dan
    penyortiran otomatis.
\end{enumerate}

\vspace{1em}

\section{Manfaat Penelitian}
\noindent
Manfaat yang diharapkan dari penelitian ini antara lain:
\begin{enumerate}
  \item Memberikan kontribusi terhadap pengembangan sistem deteksi
    objek yang cepat dan akurat di lingkungan industri, khususnya
    untuk inspeksi visual kontainer kimia dalam kondisi nyata yang bervariasi.
  \item Menyediakan solusi yang efektif dalam mendeteksi kecacatan secara
    otomatis, sehingga dapat menggantikan metode inspeksi manual yang
    memakan waktu dan rentan terhadap kesalahan manusia.
  \item Mendorong otomatisasi penuh dalam proses identifikasi dan
    penyortiran kontainer kimia, sehingga meningkatkan efisiensi
    produksi, menurunkan biaya operasional, dan meminimalkan risiko
    kesalahan klasifikasi dalam sistem manufaktur.
\end{enumerate}

\chapter{METODE PENELITIAN}
\section{Tempat dan Waktu Penelitian}
Penelitian ini dilaksanakan mulai dari bulan Februari 2025 hingga Juni
2025, bertempat di Laboratorium Elektronika dan Instrumentasi,
Departemen Fisika, Fakultas Matematika dan Ilmu Pengetahuan Alam,
Universitas Hasanuddin, Makassar.

\vspace{1em}

\section{Peralatan Penelitian}
Adapun peralatan yang digunakan pada penelitian ini adalah sebagai berikut
\begin{enumerate}
  \item Arduino Uno berf ungsi sebagai mikrokontroler utama yang
    mengendalikan motor servo pada lengan robot serta menerima sinyal
    dari sensor.
  \item Motor Servo digunakan sebagai aktuator untuk menggerakkan
    bagian-bagian lengan robot sesuai perintah dari Arduino.
  \item Lengan Robot EEZYbotARM MK1 berfungsi sebagai struktur
    mekanik yang menjadi tempat pemasangan motor servo dan berperan
    sebagai sistem pergerakan robotik.
  \item Power Supply 5V berf ungsi memberikan catu daya stabil untuk
    motor servo agar dapat beroperasi dengan baik.
  \item Sensor PIR HC-SR501 mendeteksi gerakan dan membantu
    menghitung jumlah kontainer cacat dan non-cacat yang lewat.
  \item Kamera digunakan untuk mengambil gambar kontainer kimia, yang
    kemudian diproses oleh model deteksi (YOLO) dan deteksi cacat (Autoencoder).
  \item Laptop/Komputer digunakan untuk mengunggah program ke Arduino
    Uno, serta menjalankan model deteksi berbasis YOLO dan
    Autoencoder untuk analisis visual.
  \item Kabel Jumper berfungsi menghubungkan berbagai komponen
    elektronik seperti sensor dan aktuator ke papan rangkaian dan
    Arduino.Kabel Jumper berfungs imenghubungkan setiap komponen sensor.
  \item Papan rangkaian berfungsi untuk menyediakan jalur koneksi
    antar komponen.
\end{enumerate}

\vspace{1em}

\section{Metode Kerja}
Dalam penelitian ini terdapat beberapa tahapan yang harus dilakukan,
tahapan penelitian dapat dilihat pada flowchart Gambar
\ref{fig:bagan-alir}. Penelitian ini dibatasi pada perancangan dan
pembuatan prototipe sistem deteksi cacat.
\begin{figure}[H]
  \centering
  \includegraphics[]{gambar/flowchart.jpg}
  \caption{Bagan alir penelitian}
  \label{fig:bagan-alir}
\end{figure}
\vspace{-1em}

Tahapan dimulai dengan kajian mendalam terhadap teknologi robotik,
algoritma Autoencoder untuk deteksi anomali visual, serta metode
deteksi objek seperti YOLO (You Only Look Once), guna memperoleh
pemahaman komprehensif tentang konsep dasar, format dataset, dan
teknik perancangan model untuk mendeteksi cacat pada kontainer kimia.
Setelah pemahaman awal diperoleh, dilakukan perancangan sistem yang
mencakup perangkat keras (lengan robot dan sensor) serta perangkat
lunak berupa algoritma machine learning. Selanjutnya, sistem robot
dikalibrasi agar dapat bekerja secara optimal, termasuk proses tuning
hyperparameter pada model pembelajaran mesin. Jika sistem telah
berfungsi sesuai dengan yang diharapkan, maka dilanjutkan dengan
proses pengambilan data sebagai langkah awal dalam pengujian dan
validasi model.

\vspace{1em}

\subsection{Perancangan \textit{Hardware}}
Penelitian ini dimulai dengan tahap perancangan hardware. Komponen
hardware yang digunakan meliputi kamera untuk mengambil gambar
kontainer kimia, laptop sebagai pusat pemrosesan dan eksekusi
algoritma pembelajaran mesin, serta motion sensor untuk menghitung
jumlah kontainer kimia, baik yang cacat maupun yang tidak. Adapun
rancangan hardware dapat dilihat pada Gambar \ref{fig:rangkaian}.

\begin{figure}[H]
  \centering
  \includegraphics[width=\textwidth]{gambar/rangkaian.png}
  \caption{Bagan alir penelitian}
  \label{fig:rangkaian}
\end{figure}
\vspace{-1em}

Sistem diawali saat kamera menangkap gambar kontainer kimia yang
diletakkan di area pengambilan gambar. Gambar ini diproses oleh model
deteksi objek (YOLO) untuk mengenali keberadaan kontainer sebelum
tahap deteksi cacat. Selanjutnya, gambar yang telah dikenali dikirim
ke model deteksi kecacatan berbasis Convolutional Autoencoder untuk
menentukan apakah kontainer mengalami cacat atau tidak. Berdasarkan
hasil analisis tersebut, sinyal dikirimkan ke mikrokontroler Arduino
untuk menggerakkan servo sebagai respon terhadap kondisi kontainer.
Lengan robot kemudian mengambil kontainer kimia dan memindahkannya ke
wadah yang sesuai, tergantung pada hasil deteksi—apakah kontainer
tersebut cacat atau tidak. Untuk memantau dan menghitung jumlah
kontainer yang telah dipindahkan, dua motion sensor dipasang pada
masing-masing wadah (cacat dan noncacat). Data dari sensor ini
dikirim ke web server, yang menyediakan API untuk dikonsumsi agar
data jumlah kontainer dapat ditampilkan ke klien secara real-time.
Secara keseluruhan, keterkaitan antar komponen perangkat keras dalam
sistem ini dapat dilihat pada Gambar \ref{fig:hardware}.

\begin{figure}[H]
  \centering
  \includegraphics[width=\textwidth]{gambar/rancang.png}
  \caption{Rancang sistem \textit{hardware}}
  \label{fig:hardware}
\end{figure}
\vspace{-1em}

\vspace{1em}

\subsection{Perancangan \textit{Software}}
Perangkat lunak dalam penelitian ini mencakup perancangan beberapa
algoritma inti. Pertama, dirancang model deteksi objek berbasis YOLO
untuk mengenali kontainer kimia pada citra yang diambil oleh kamera.
Kedua, digunakan Convolutional Autoencoder sebagai model untuk
mendeteksi cacat atau anomali visual pada kontainer. Selain itu,
dirancang pula algoritma kontrol untuk mengatur pergerakan lengan
robot dalam mengambil dan memindahkan kontainer berdasarkan hasil
klasifikasi. Sistem ini juga terintegrasi dengan modul IoT untuk
menampilkan data kontainer cacat dan non-cacat pada klien secara real-time. \par

Tahap perancangan model deteksi objek dimulai dengan pengumpulan
dataset berupa gambar kontainer kimia dari berbagai kondisi dan sudut
pandang menggunakan kamera, yang nantinya dipasang bersama lengan
robot. Setelah gambar terkumpul, dilakukan proses anotasi dengan
memberikan label dan bounding box pada setiap kontainer sesuai dengan
format yang dibutuhkan oleh algoritma YOLO. Dataset yang telah
dianotasi kemudian digunakan untuk melatih model deteksi objek.
Tujuannya agar model mampu mendeteksi kontainer kimia secara akurat
dan cepat dalam berbagai situasi, misalnya ketika kontainer berada
dalam posisi miring. Setelah proses pelatihan selesai, model
dievaluasi menggunakan data uji yang belum pernah dilihat sebelumnya.
Evaluasi dilakukan menggunakan beberapa metrik seperti precision,
recall, dan mean Average Precision (mAP), guna memastikan bahwa model
memiliki kemampuan generalisasi yang baik dan layak diterapkan di
sistem robotik secara real-time. Pipeline perancangan model deteksi
objek dapat dilihat pada Gambar \ref{fig:pipeline-yolo}.

\begin{figure}[H]
  \centering
  \includegraphics[width=\textwidth]{gambar/pipeline_yolo.png}
  \caption{Diagram \textit{pipeline} pelatihan model YOLO}
  \label{fig:pipeline-yolo}
\end{figure}
\vspace{-1em}

Berikutnya adalah tahap pembangunan model deteksi cacat menggunakan
algoritma Convolutional Autoencoder. Tahap ini menggunakan dataset
yang sama seperti yang digunakan pada pelatihan model YOLO, namun
tanpa menggunakan anotasi bounding box karena sifat unsupervised dari
autoencoder. Data diproses melalui tahap preprocessing seperti
resizing, normalisasi, dan augmentasi (rotasi, flipping, pencahayaan)
untuk meningkatkan variasi. Model dirancang dengan dua komponen
utama: encoder untuk mengekstraksi fitur penting dan menghasilkan
representasi berdimensi rendah (latent space), serta decoder untuk
merekonstruksi gambar dari representasi tersebut. Setelah arsitektur
selesai dan data siap, model dilatih untuk meminimalkan perbedaan
antara gambar asli dan hasil rekonstruksi, sehingga mampu mengenali
citra normal secara akurat. Evaluasi dilakukan dengan menghitung
reconstruction error, yang digunakan untuk membedakan antara gambar
normal dan cacat. Ambang batas deteksi ditentukan melalui analisis
distribusi error pada data validasi. Pipeline perancangan model
deteksi cacat dapat dilihat pada Gambar \ref{fig:pipeline-autoencoder}.

\begin{figure}[H]
  \centering
  \includegraphics[width=\textwidth]{gambar/pipeline_autoencoder.png}
  \caption{Diagram \textit{pipeline} pelatihan model deteksi cacat}
  \label{fig:pipeline-autoencoder}
\end{figure}
\vspace{-1em}

\vspace{1em}

\subsection{Bagan Alir Sistem Kerja Alat}
Bagan alir sistem kerja alat terdiri dari 2 bagian yakni sistem slot
parkir dan sistem gerbang parkir. Perancangan bagan alir sistem kerja
alat secara keseluruhan ditunjukkan pada Gambar \ref{fig:bagan-alir-kerja}.

\begin{figure}[H]
  \centering
  \includegraphics[]{gambar/flowchart.jpg}
  \caption{Bagan alir sistem kerja alat}
  \label{fig:bagan-alir-kerja}
\end{figure}
\vspace{-1em}

\chapter{HASIL DAN PEMBAHASAN}
\section{Hasil Perancangan Sistem}
Sistem deteksi kecacatan kontainer yang dikembangkan pada penelitian
ini dirancang untuk bekerja secara otomatis dengan menggunakan kamera
sebagai sensor utama dan lengan robot sebagai aktuator. Proses
diawali oleh model YOLO yang mendeteksi keberadaan kontainer dari
citra hasil tangkapan kamera. Jika objek kontainer berhasil
terdeteksi, citra tersebut dianalisis oleh model \textit{autoencoder}
untuk mengklasifikasikan apakah kontainer mengalami kecacatan.
Berdasarkan hasil klasifikasi, mikrokontroler mengendalikan motor
\textit{servo} untuk menggerakkan lengan robot guna menyortir
kontainer ke dalam kategori cacat atau tidak cacat. Secara paralel,
sensor PIR digunakan untuk menghitung jumlah total kontainer pada
masing-masing kategori. Data ini kemudian dikirim (melalui metode
POST) ke \textit{web server} untuk divisualisasikan pada sisi klien
melalui tampilan \textit{website}. Desain sistem yang telah dirancang
diilustrasikan pada Gambar \ref{fig:rancang-sistem}.

\begin{figure}[H]
  \centering
  \includegraphics[width=\textwidth]{gambar/rancang_sistem.jpg}
  \caption{Hasil perancangan sistem}
  \label{fig:rancang-sistem}
\end{figure}
\vspace{-1em}

\vspace{1em}

\section{Hasil Perancangan Model YOLO}
\subsection{\textit{Dataset} dan \textit{Preprocessing}}
Tahap awal penelitian adalah pengumpulan data primer berupa citra
kontainer kimia menggunakan kamera. Proses akuisisi ini dilakukan
untuk membangun  sebuah \textit{dataset} kustom yang mampu merepresentasikan
objek target secara akurat. Total citra mentah yang berhasil
dikumpulkan berjumlah 463 citra yang diambil dari berbagai posisi dan
sudut pandang untuk memastikan model dapat mengenali objek dalam
berbagai kondisi. \textit{Dataset} kemudian dibagi menjadi 395 citra
untuk data latih dan 68 citra data validasi. Distribusi pembagian dataset
disajikan pada Tabel \ref{tab:pembagian-dataset}.

\begin{table}[H]
  \caption{Distribusi pembagian \textit{dataset}}
  \label{tab:pembagian-dataset}
  \vspace{-1em}
  \centering
  \begin{tabular}{ccc}
    \toprule
    \textbf{Kategori} & \textbf{Jumlah Gambar} & \textbf{Persentase} \\
    \midrule
    Data Latih & 395 & 85,3\% \\
    Data Validasi & 68 & 14,7\% \\
    Total Data & 463 & 100\% \\
    \bottomrule
  \end{tabular}
\end{table}

Setelah akuisisi data, dilakukan anotasi citra untuk
memberikan \textit{bounding box} dan label kelas pada setiap objek yang
terdeteksi di dalam citra. Proses ini penting karena arsitektur YOLO
dirancang untuk secara simultan memprediksi letak objek dan kelasnya,
sehingga memerlukan data berlabel \citep{19,20}. Sebanyak 463 citra
kontainer telah dianotasi secara manual. Contoh hasil anotasi dapat
dilihat pada Gambar \ref{fig:yolo-anotasi}.

\begin{figure}[H]
  \centering
  \includegraphics[width=0.7\textwidth]{gambar/anotasi.png}
  \caption{Salah satu data \textit{train} dengan labelnya}
  \label{fig:yolo-anotasi}
\end{figure}
\vspace{-1em}

\vspace{1em}

\subsection{Hasil Pelatihan Model}
Dalam penelitian ini, metrik utama yang digunakan untuk mengevaluasi
performa model YOLO adalah mAP. Metrik ini penting untuk menilai
seberapa akurat model dalam mendeteksi objek
seingga lengan robot dapat melakukan penyortiran secara tepat. Nilai
mAP dihitung berdasarkan rata-rata \textit{Average Precision} (AP), yang
merupakan gabungan antara \textit{precision} dan \textit{recall}
untuk setiap kelas \citep{21}.

Untuk menentukan apakah sebuah prediksi tergolong benar (\textit{True
Positive}) atau salah (\textit{False Positive}), digunakan metrik
\textit{Intersection over Union} (IoU). IoU yaitu rasio antara
\textit{bounding box} prediksi ($BB_{predict}$) dengan
\textit{bounding box} sebenarnya ($BB_{ground}$), dengan nilai
mendekati 1 menunjukkan prediksi yang sangat akurat \citep{22}.
Semakin mendekati 1, berarti prediksi semakin akurat dan sesuai
dengan objek sebenarnya. IoU dihitung menggunakan persamaan:
\begin{equation}
  IoU = \frac{|BB_{predict} \cap
  BB_{ground}|}{|BB_{predict} \cup BB_{ground}|}
\end{equation}
\indent
Selanjutnya, \textit{precision} mengukur proporsi prediksi positif
yang benar (\textit{True Positive}) terhadap seluruh prediksi positif
(TP + \textit{False Positive}), sedangkan \textit{recall} atau
\textit{true positive rate} mengukur seberapa banyak objek yang
benar-bernar terdeteksi \citep{23}. \textit{Precision} dan
\textit{recall} dapat dihitung menggunakan persamaan:
\begin{equation}
  Precision = \frac{TP}{TP + FP}, \quad
  Recall = \frac{TP}{TP + FN}
\end{equation}
\indent
Karena model YOLO dalam penelitian ini hanya memprediksi satu kelas
yaitu kontainer kimia, maka nilai mAP setara dengan nilai AP. Nilai
AP dihitung sebagai rata-rata \textit{precision} di seluruh rentang nilai
\textit{recall} (0 hingga 1) \citep{24}, sebagaimana dirumuskan dalam
Persamaan 3, di mana P adalah \textit{precision} dan r adalah \textit{recall}.
\begin{equation}
  AP = \int_{0}^{1} P(r) \,dr
\end{equation}
\indent
Proses pelatihan model YOLO dilakukan di platform \textit{Google
Colaboratory} selama 50 \textit{epoch} menggunakan \textit{batch
size} 16 dan \textit{optimizer} Adam. Ringkasan hasil pelatihan
setiap 10 \textit{epoch} disajikan pada Tabel \ref{tab:yolo-train}.
\begin{table}[H]
  \caption{Proses \textit{training} model YOLO}
  \label{tab:yolo-train}
  \vspace{-1em}
  \centering
  \small
  \begin{tabular}{c p{1.5cm} p{1.5cm} p{1.5cm} p{1.5cm} p{1.5cm} p{1.5cm}}
    \toprule
    \textbf{Epoch} & \textbf{Train/Box Loss} & \textbf{Train/Class Loss}
    & \textbf{Train/DFL Loss} & \textbf{Val/Box Loss}
    & \textbf{Val/Class Loss} & \textbf{Val/DFL Loss} \\
    \midrule
    0  & 0,5946 & 1,9374 & 0,9706 & 0,4709 & 2,5029 & 0,8990 \\
    10 & 0,4333 & 0,4356 & 0,8713 & 0,3052 & 0,3758 & 0,8262 \\
    20 & 0,3286 & 0,2935 & 0,8510 & 0,3342 & 0,2105 & 0,8448 \\
    30 & 0,2821 & 0,2328 & 0,8390 & 0,2418 & 0,1566 & 0,8244 \\
    40 & 0,2267 & 0,1776 & 0,8010 & 0,2209 & 0,1386 & 0,8201 \\
    50 & 0,2016 & 0,1546 & 0,7993 & 0,2003 & 0,1163 & 0,8156 \\
    \bottomrule
  \end{tabular}
  \normalsize
\end{table}
Nilai \textit{loss} pada data latih dan validasi secara umum
menunjukkan tren menurun, mengindikasikan bahwa model berhasil
menyesuaikan parameter internalnya dengan data. Pada
\textit{Train/Box Loss}, terjadi penurunan
signifikan dari 0,5946 pada \textit{epoch} ke-0 menjadi 0,2016 pada
\textit{epoch} ke-50, yang menunjukkan peningkatan akurasi deteksi
posisi objek. Penurunan signifikan juga terlihat pada \textit{Train/Class
Loss} dan \textit{Train/DFL Loss}, serta pada seluruh komponen
\textit{loss} di data validasi. Penurunan konsisten ini menunjukkan
bahwa model memiliki kemampuan generalisasi yang
baik pada data validasi.

Meskipun demikian, nilai \textit{loss} saja tidak cukup untuk menilai
keseluruhan performa model. Oleh karena itu, digunakan pula metrik
mAP, seperti mAP@50 dan mAP@50-95, untuk
mengevaluasi ketepatan deteksi: mAP@50 berarti evaluasi dilakukan
dengan ambang IoU 50\% yang cenderung lebih longgar, sedangkan
mAP@50-95 adalah rata-rata mAP pada ambang IoU dari 50\% hingga 95\%
dengan interval 5\%, sehingga lebih ketat. Hasil pengukuran mAP pada
tiap \textit{epoch} dapat dilihat pada Gambar \ref{fig:map}.

\begin{figure}[H]
  \centering
  % First image
  \begin{minipage}[]{\textwidth}
    \centering
    \includegraphics[width=0.625\textwidth]{gambar/map50.png}
    (a)
  \end{minipage}
  \vspace{1em}

  % Second image
  \begin{minipage}{\textwidth}
    \centering
    \includegraphics[width=0.625\textwidth]{gambar/map5090.png}
    (b)
  \end{minipage}
  \caption{Grafik tren mAP: (a) mAP@50, (b) mAP@50-95}
  \label{fig:map}
  \vspace{-1em}
\end{figure}

Grafik (a) menunjukkan bahwa model dengan cepat mencapai nilai mAP
tinggi pada ambang 50\%, bahkan dalam 10-15 \textit{epoch} pertama.
Grafik (b) menunjukkan tren yang lebih lambat dan bertahap,
mengindikasikan proses penyempurnaan presisi dan akurasi posisi
\textit{bounding box}. Secara keseluruhan, nilai mAP yang tinggi pada
kedua metrik diakhir
pelatihan menunjukkan bahwa model yang dihasilkan tidak hanya mampu
mendeteksi objek, tetapi juga mampu menentukan batas objek secara akurat.
Contoh hasil prediksi model YOLO terhadap data validasi disajikan
pada Gambar \ref{fig:yolo-validasi}.

\begin{figure}[H]
  \centering
  \includegraphics[width=0.93\textwidth]{gambar/yolo_validasi.jpg}
  \caption{Hasil prediksi YOLO pada data validasi}
  \label{fig:yolo-validasi}
\end{figure}
\vspace{-1em}

\vspace{1em}

\section{Hasil Perancangan Model Deteksi Kecacatan}
\subsection{Arsitektur Variational Autoencoder}
\textit{Variational autoencoder} (VAE) merupakan pengembangan dari
\textit{autoencoder}, yaitu arsitektur \textit{neural network}
yang berfungsi untuk mengekstraksi fitur laten dari data. Perbedaan
utama VAE dibandingkan \textit{autoencoder} konvensional adalah
pendekatannya yang berbasis Bayesian, di mana vektor laten ($z$)
dipaksa mengikuti distribusi probabilitas terstruktur, biasanya
distribusi Gaussian. Secara operasional, bagian \textit{encoder} VAE
($q_{\phi}(z \mid x)$) tidak hanya mengompresi data masukan ($x$),
tetapi juga memetakannya ke dalam parameter statistik: yaitu vektor
rata-rata (\textit{mean}) dan vektor varians (\textit{log-varians}).
Kemudian, sebuah vektor laten ($z$) diambil dari distribusi dan
diteruskan ke \textit{decoder} ($p_{\theta}(x \mid z)$) untuk
merekonstruksi kembali data masukan \citep{25}. Arsitektur
\textit{variational autoencoder} yang
digunakan pada penelitian ini dapat dilihat pada Gambar
\ref{fig:arsitektur-autoencoder}.

\begin{figure}[H]
  \centering
  \includegraphics[width=\textwidth]{gambar/arsitektur_autoencoder.png}
  \caption{Arsitektur \textit{variational autoencoder}}
  \label{fig:arsitektur-autoencoder}
\end{figure}
\vspace{-1em}

Arsitektur VAE dirancang untuk memproses citra berukuran 128 $\times$
128 dengan tiga kanal warna (RGB). Bagian \textit{encoder}
terdiri atas lima lapisan
\textit{convolutional} berurutan, masing-masing diikuti fungsi
aktivasi ReLU yang bertugas menurunkan
resolusi fitur spasial dan mengekstraksi fitur penting hingga
menghasilkan representasi akhir berdimensi (512, 4, 4). Representasi
ini kemudian diproyeksikan menjadi dua vektor statistik, yaitu vektor
\textit{mean} dan vektor \textit{log-varians}, yang bersama-sama
membentuk distribusi
Gaussian multivariat sebagai ruang laten. Setelah proses pengambilan
sampel dari ruang laten, vektor z diteruskan ke \textit{decoder} yang
memproyeksikan kembali representasi tersebut melalui lapisan linear
dan beberapa lapisan \textit{transpose convolutional} (dekonvolusi) untuk
mengembalikan ukuran citra ke dimensi semula. Fungsi aktivasi
sigmoid pada lapisan \textit{output} memastikan nilai piksel berada pada
rentang [0, 1], sehingga citra hasil rekonstruksi dapat
diinterpretasikan sebagai citra valid.

\vspace{1em}

\subsection{\textit{Dataset} dan \textit{Preprocessing}}
Tahap ini diawali dengan pengumpulan data primer berupa citra
kontainer kimia menggunakan kamera. Tujuannya adalah membangun
\textit{dataset} kustom yang dapat merepresentasikan objek target secara
spesifik. Total citra mentah yang berhasil dikumpulkan berjumlah 3005
citra, yang seluruhnya digunakan sebagai data latih. Sebelum
digunakan untuk pelatihan \textit{autoencoder}, seluruh citra diproses
terlebih dahulu menggunakan model YOLO untuk mendeteksi serta
memotong (\textit{cropping}) area kontainer secara
otomatis. Hal ini bertujuan agar hanya bagian penting dari kontainer
yang diproses lebih lanjut, sehingga model \textit{autoencoder} dapat
secara optimal mempelajari karakteristik visual kontainer. Contoh
salah satu citra data latih ditunjukkan pada Gambar \ref{fig:ae-data}.

\begin{figure}[H]
  \centering
  \includegraphics[width=\textwidth]{gambar/ae_data.png}
  \caption{Salah satu data latih model \textit{autoencoder}}
  \label{fig:ae-data}
\end{figure}
\vspace{-1em}

Karena \textit{autoencoder} termasuk dalam kategori
\textit{unsupervised learning},
maka data tidak memerlukan label atau anotasi manual. \textit{Dataset} ini
hanya terdiri dari citra kontainer dalam kondisi baik (tanpa
kerusakan/cacat visual), sehingga model \textit{autoencoder} dapat mempelajari
distribusi visual normal secara optimal. Dengan demikian, saat
digunakan pada data yang mengandung cacat model akan menunjukkan
perbedaan yang mencolok antara citra asli dan hasil rekonstruksi yang
menjadi dasar untuk mendeteksi anomali.

\vspace{1em}

\subsection{Hasil Pelatihan Model}
Fungsi \textit{loss} pada VAE didasarkan pada prinsip
\textit{Evidence Lower Bound}
(ELBO) yang bertujuan mengoptimalkan model probabilistik dengan
mengatasi kendala distribusi posterior yang tidak dapat dihitung
secara langsung. Fungsi \textit{loss} VAE terdiri dari dua komponen utama: (1)
\textit{reconstruction loss}, yang mengukur selisih antara citra asli (x) dan
citra hasil rekonstruksi dari vektor laten z, dan (2) \textit{regularization
term} berupa divergensi Kullback-Leibler (KL) yang mengukur sejauh
mana distribusi posterior hasil \textit{encoder} menyimpang dari distribusi
Gaussian standar \citep{26}. Fungsi \textit{loss} VAE dirumuskan
sebagai berikut:
\begin{equation}
  \mathcal{L}(\theta, \phi; x) = \mathbb{E}{q\phi(z|x)}[\log
  p_\theta(x|z)] - D_{KL}(q_\phi(z|x) \parallel p_\theta(z))
\end{equation}
\indent
Model CVAE dilatih
menggunakan komputer dengan GPU NVIDIA GeForce RTX 3050 8GB VRAM
untuk memanfaatkan kemampuan komputasi paralel CUDA. Proses pelatihan
dilakukan selama 100 \textit{epoch} dengan
\textit{batch size} 64
dan \textit{optimizer} Adam. Nilai \textit{loss} setiap 10
\textit{epoch} ditampilkan pada Tabel \ref{tab:training-autoencoder}.

\begin{table}[H]
  \caption{Proses \textit{training} model model \textit{convolutional
  variational autoencoder}}
  \label{tab:training-autoencoder}
  \vspace{-1em}
  \centering
  \begin{tabular}{cc}
    \toprule
    \textbf{Epoch} & \textbf{Loss} \\
    \midrule
    1 & 837,7615 \\
    10 & 30,5440 \\
    20 & 22,6941 \\
    30 & 18,5220 \\
    40 & 15,0172 \\
    50 & 11,9098 \\
    60 & 9,8593 \\
    70 & 8,7772 \\
    80 & 7,3923 \\
    90 & 7,1070 \\
    100 & 6,8338 \\
    \bottomrule
  \end{tabular}
\end{table}
Penurunan nilai \textit{loss} selama proses pelatihan menunjukkan bahwa model
CVAE secara bertahap berhasil mempelajari pola visual dari kontainer
normal. Dengan semakin kecilnya nilai \textit{loss}, model menunjukkan
kemampuan yang semakin baik dalam merekonstruksi citra tanpa cacat,
sehingga mampu membedakan citra cacat secara signifikan. Hasil
rekonstruksi ditampilkan pada Gambar \ref{fig:autoencoder-test}.

\begin{figure}[H]
  \centering
  % First image
  \begin{minipage}{\textwidth}
    \centering
    \includegraphics[width=\textwidth]{gambar/kontainer_bagus.jpeg}
    (a)
  \end{minipage}
  \vspace{1em}

  % Second image
  \begin{minipage}{\textwidth}
    \centering
    \includegraphics[width=\textwidth]{gambar/kontainer_cacat.jpeg}
    (b)
  \end{minipage}
  \caption{Hasil prediksi kecacatan: (a) Kontainer normal, (b)
  kontainer cacat}
  \label{fig:autoencoder-test}
  \vspace{-1em}
\end{figure}

Dari gambar di atas, diketahui bahwa nilai rekonstruksi
\textit{error} untuk kontainer
normal adalah 0,006447, sedangkan untuk kontainer cacat sebesar
0,007566. Selisih ini menunjukkan bahwa model mampu membedakan citra
cacat dari normal normal berdasarkan tingkat kesalahan rekonstruksi.

\vspace{1em}

\subsection{Penentuan Ambang Batas Kecacatan}
Penentuan ambang batas merupakan langkah penting dalam memastikan
sistem deteksi cacat dapat secara akurat membedakan kontainer cacat
dan tidak cacat. Tanpa ambang batas yang tepat, sistem beresiko
menghasilkan banyak kesalahan klasifikasi yang tinggi. Dalam
penelitian ini, digunakan metrik \textit{Mean Squared Error} (MSE) untuk
mengukur perbedaan antara citra asli dan hasil rekonstruksi.

MSE menghitung rata-rata selisih kuadrat antara piksel citra asli
(\textit{input}) dengan piksel citra hasil rekonstruksi
(\textit{output}). Semakin
kecil nilai MSE, semakin mirip citra hasil rekonstruksi dengan citra
aslinya, yang berarti model berhasil meminimalkan distorsi
\citep{27}. MSE dirumuskan sebagai berikut:

\begin{equation}
  MSE = \frac{1}{n} \sum_{i=1}^{n} (x_i - \hat{x}_i)^2
\end{equation}

Pengujian dilakukan terhadap 20 citra, terdiri dari 10 kontainer
normal dan 10 gambar kontainer cacat. Dari pengujian ini diperoleh
distribusi nilai MSE masing-masing citra untuk dianalisis lebih
lanjut. Hasil perhitungan MSE pada 20 gambar uji tersebut dirangkum
pada Tabel \ref{tab:error-samples}.

\begin{table}[H]
  \centering
  \caption{Nilai \textit{error} untuk setiap sampel}
  \label{tab:error-samples}
  \begin{tabular}{ccc}
    \toprule
    \textbf{No} & \textbf{Error} & \textbf{Kategori} \\
    \midrule
    1  & 0,006737 & Normal \\
    2  & 0,006828 & Normal \\
    3  & 0,007091 & Normal \\
    4  & 0,006997 & Normal \\
    5  & 0,006610 & Normal \\
    6  & 0,006795 & Normal \\
    7  & 0,006872 & Normal \\
    8  & 0,006725 & Normal \\
    9  & 0,006873 & Normal \\
    10 & 0,006817 & Normal \\
    11 & 0,007432 & Cacat \\
    12 & 0,007455 & Cacat \\
    13 & 0,008369 & Cacat \\
    14 & 0,007183 & Cacat \\
    15 & 0,007653 & Cacat \\
    16 & 0,008856 & Cacat \\
    17 & 0,007531 & Cacat \\
    18 & 0,007624 & Cacat \\
    19 & 0,007457 & Cacat \\
    20 & 0,008424 & Cacat \\
    \bottomrule
  \end{tabular}
\end{table}

Dari tabel di atas, nilai MSE untuk kontainer normal berkisar 0,0067
sedangkan nilai untuk kontainer cacat secara konsisten lebih tinggi.
Hal ini menunjukkan bahwa perbedaan rekonstruksi pada citra cacat
lebih besar, menandakan kegagalan model untuk merekonstruksi fitur
yang tidak dikenalnya. Untuk mnentukan ambang batas optimal,
digunakan metode \textit{Receiver Operating Characteristic} (ROC)
yang memplot \textit{true positive rate} (recall) terhadap
\textit{false positive rate} (1 – \textit{specificity}) \citep{28}.
Nilai ambang terbaik ditentukan dengan memaksimalkan statistik Youden
J yang dirumuskan sebagai:

\begin{equation}
  J = \textit{True Positive Rate} - \textit{False Positive Rate}
\end{equation}

Kurva ROC dari hasil pengujian ditampilkan pada Gambar \ref{fig:roc}
digunakan sebagai metode analisis untuk mengevaluasi performa model
deteksi cacat pada kontainer kimia dalam penelitian ini. Kurva ROC dibuat
dengan memplot nilai 1 - \textit{specificity} (\textit{false positive
rate}) pada sumbu-x dan \textit{recall} (\textit{true positive rate})
pada sumbu-y untuk setiap nilai ambang batas yang diuji.
\textit{Specificity} sendiri adalah ukuran seberapa baik model dalam
mengenali data negatif secara benar. Nilai ambang batas terbaiki
ditentukan dengan memaksimalkan statistik Youden J, yang dirumuskan
pada Persamaan 6.

\begin{figure}[H]
  \centering
  \includegraphics[width=\textwidth]{gambar/roc.png}
  \caption{Kurva ROC}
  \label{fig:roc}
\end{figure}
\vspace{-1em}

Kurva ROC menunjukkan bahwa model memiliki performa yang sangat baik,
dengan garis mendeteksi sudut kiri atas. Berdasarkan kurva, diperoleh
nilai ambang optimal ditunjukkan oleh garis biru yang mendekati titik
sudut kiri atas. Berdasarkan perhitungan, ambang batas optimal
sebesar 0,007183, ditandai dengan lingkaran merah pada grafik. Nilai
ini digunakan sebagai batas akhir dalam klasifikasi kontainer cacat
dan tidak cacat, guna memaksimalkan sensitivitas sekaligus
meminimalkan kesalahan deteksi positif.

\vspace{1em}

\section{Hasil Pengujian Sistem Lengan Robot}

Setelah model deteksi terlatih dan nilai ambang batas optimal
ditentukan, tahap selanjutnya adalah integrasi sistem dengan lengan
robot yang dikendalikan oleh mikrokontroler Arduino. Berdasarkan
hasil klasifikasi dari model, lengan robot  akan secara otomatis
memindahkan kontainer kimia. Jika objek teridentifikasi sebagai
cacat, maka lengan robot akan mengarahkan ke sisi kiri, sedangkan
jika dinyatakan normal akan diarahkan ke sisi kanan . Gambar
\ref{fig:robot-only} mengilustrasikan aksi lengan robot
ketika sedang menyortir objek yang teridentifikasi normal dan cacat.

\begin{figure}[H]
  \centering
  % First image
  \begin{minipage}{\textwidth}
    \centering
    \includegraphics[width=0.88\textwidth]{gambar/robot_normal.jpeg}\\
    (a)
  \end{minipage}
  \vspace{1em}

  % Second image
  \begin{minipage}{\textwidth}
    \centering
    \includegraphics[width=0.88\textwidth]{gambar/robot_cacat.jpeg}\\
    (b)
  \end{minipage}
  \caption{Lengan robot menyortir kontainer: (a) normal, (b) cacat}
  \label{fig:robot-only}
  \vspace{-1em}
\end{figure}

Setelah lengan robot melepaskan kontainer di area penyortiran yang
telah ditentukan, sebuah sensor PIR yang terpasang di lokasi
tersebut akan mendeteksi keberadaan objek. Deteksi dari sensor PIR
ini berfungsi sebagai sinyal konfirmasi bahwa satu siklus penyortiran
telah selesai. Sinyal ini kemudian memicu pengiriman data hasil
klasifikasi, yang mencakup citra kontainer beserta label statusnya,
ke \textit{web server}. Untuk menampilkan informasi ini secara
\textit{real-time} kepada klien, sistem memanfaatkan protokol
\textit{WebSocket}. Tampilan \textit{website} dan hasil prediksi
dapat dilihat pada Gambar \ref{fig:web-test}.

\begin{figure}[H]
  \centering
  % First image
  \begin{minipage}{\textwidth}
    \centering
    \includegraphics[width=\textwidth]{gambar/web_ss_normal.png}
    (a)
  \end{minipage}
  \vspace{1em}

  % Second image
  \begin{minipage}{\textwidth}
    \centering
    \includegraphics[width=\textwidth]{gambar/web_ss_cacat.png}
    (b)
  \end{minipage}
  \centering
  \caption{Tampilan hasil prediksi melalui \textit{website}: (a)
  kontainer normal, (b) kontainer cacat}
  \label{fig:web-test}
  \vspace{-1em}
\end{figure}

Untuk menguji keandalan dan konsistensi performa sistem, dilakukan
pengujian sebanyak 25 kali menggunakan sampel kontainer dengan
kondisi acak (normal dan cacat). Hasil kuantitatif dari setiap
pengujian ini dirangkum secara pada Tabel \ref{tab:error-samples-web}.

\begin{table}[H]
  \centering
  \caption{Nilai \textit{error} untuk setiap sampel uji}
  \label{tab:error-samples-web}
  \begin{tabular}{cccc}
    \toprule
    \textbf{No} & \textbf{Error} & \textbf{Kategori
      (Ambang Batas <
    0,007183)} & \textbf{Label Asli} \\
    \midrule
    1  & 0,006798 & Normal & Normal \\
    2  & 0,008298 & Cacat & Cacat \\
    3  & 0,006854 & Normal & Normal \\
    4  & 0,007066 & Normal & Normal \\
    5  & 0,006923 & Normal & Normal \\
    6  & 0,007607 & Cacat & Cacat \\
    7  & 0,007261 & Cacat & Cacat \\
    8  & 0,007393 & Cacat & Cacat \\
    9  & 0,006842 & Normal & Normal \\
    10 & 0,008099 & Cacat & Cacat \\
    11 & 0,006945 & Normal & Normal \\
    12 & 0,007381 & Cacat & Cacat \\
    13 & 0,006722 & Normal & Normal \\
    14 & 0,006865 & Normal & Normal \\
    15 & 0,007397 & Cacat & Cacat \\
    16 & 0,007099 & Normal & Normal \\
    17 & 0,006790 & Normal & Normal \\
    18 & 0,006760 & Normal & Normal \\
    19 & 0,007383 & Cacat & Cacat \\
    20 & 0,007693 & Cacat & Cacat \\
    21 & 0,008071 & Cacat & Cacat \\
    22 & 0,007977 & Cacat & Cacat \\
    23 & 0,006885 & Normal & Normal \\
    24 & 0,008007 & Cacat & Cacat \\
    25 & 0,006847 & Normal & Normal \\
    \midrule
    \multicolumn{3}{r}{\textbf{Tingkat Akurasi}} & \textbf{100\%} \\
    \bottomrule
  \end{tabular}
\end{table}

Hasil klasifikasi ditentukan berdasarkan perbandingan antara \textit{error}
rekonstruksi dengan ambang batas 0,007183. Sampel dengan \textit{error} lebih
rendah dikategorikan sebagai “Normal”, sedangkan yang lebih tinggi
dikategorikan sebagai “Cacat”. Dari Tabel 5 terlihat bahwa seluruh
prediksi model sesuai dengan kondisi sebenarnya (label asli),
menghasilkan tingkat akurasi 100\%. Contohnya, sampel no. 2 memiliki
nilai \textit{error} yang tinggi (0,008298) terklarifikasi dengan benar
”Cacat”. Sebaliknya sampel no. 1, dengan \textit{error} yang rendah
(0,006798) teklarifikasi dengan benar sebagai ”Normal”. Hasil
prediksi akhir pada \textit{website} ditunjukkan pada Gambar
\ref{fig:web-25}.

\begin{figure}[H]
  \centering
  \includegraphics[width=\textwidth]{gambar/ss_web_25.png}
  \caption{Hasil prediksi kecacatan pada 25 sampel}
  \label{fig:web-25}
\end{figure}
\vspace{-1em}

\chapter{KESIMPULAN}
\section{Kesimpulan}
Berdasarkan hasil perancangan, analisis, dan implementasi alat yang telah digunakan, maka dapat diambil kesimpulan sebagai berikut:
\begin{enumerate}
   \item Kesimpulan 1
   \item Kesimpulan 2
\end{enumerate}

\vspace{1em}

\section{Saran}
Alat yang telah dibuat memiliki potensi untuk dikembangkan lebih lanjut. Berikut adalah beberapa saran untuk penelitian selanjutnya:
\begin{enumerate}
   \item Saran 1
   \item Saran 2
\end{enumerate}


\addcontentsline{toc}{chapter}{DAFTAR PUSTAKA}

\begin{thebibliography}{9}

  \bibitem[Jhang et al.(2024)]{1}
  Jhang, J. Y. \& Lin, C. J. 2024. Jhang, Jyun-Yu, \& Cheng-Jian
  Lin. Optimizing parameters of YOLO model through uniform
  experimental design for gripping tasks performed by an internet of
  things–based robotic arm. Internet of Things 27, 1-12. doi:
  10.1016/j.iot.2024.101332.

  \bibitem[Oaki et al.(2023)]{2}
  Oaki, J., Sugiyama, N., Ishihara, Y., Ooga, J., Kano, H. \& Ohno,
  H., 2023. Micro-Defect Inspection on Curved Surface Using a 6-DOF
  Robot Arm with One-Shot BRDF Imaging. IFAC-PapersOnLine 56(2),
  9354-9359. doi: 10.1016/j.ifacol.2023.10.224.

  \bibitem[Lin et al.(2024)]{3}
  Lin, C. J., Jhang, J. Y., Gao, Y. J. \& Huang, H. M., 2024.
  Vision-based Robotic Arm in Defect Detection and Object
  Classification Applications. Sensors \& Materials 36(2), 655-670.
  doi: 10.18494/SAM4683.

  \bibitem[Truong et al.(2024)]{4}
  Truong, A. M. \& Luong, H. Q., 2024. A non-destructive,
  autoencoder-based approach to detecting defects and contamination
  in reusable food packaging. Current Research in Food Science 8, 1-12.
  doi: 10.1016/j.crfs.2024.100758.

  \bibitem[Graetz et al.(2018)]{5}
  Graetz, G. \& Michaels, G., 2018. Robots at work. Review of
  economics and statistics 100(5), 753-768. doi: 10.1162/rest\_a\_00754.

  \bibitem[Gihleb et al.(2022)]{6}
  Gihleb, R., Giuntella, O., Stella, L. \& Wang, T., 2022.
  Industrial robots, workers’ safety, and health. Labour economics
  78, 1-12. doi: 10.1016/j.labeco.2022.102205.

  \bibitem[Zhuo dan Zhao(2025)]{29}
  Zhou, Y., \& Zhao, Z., 2025. MPA-YOLO: Steel Surface Defect
  Detection Based on Improved YOLOv8 Framework. Pattern Recognition
  168, 1-11. doi: 10.1016/j.patcog.2025.111897.

  \bibitem[Dong et al.(2025)]{30}
  Dong, L., Zhu, H., Ren, H., Lin, T. Y., \& Lin, K. P., 2025. A
  novel lightweight MT-YOLO detection model for identifying defects
  in permanent magnet tiles of electric vehicle motors. Expert
  Systems with Applications 288, 1-13. doi: 10.1016/j.eswa.2025.128247.

  \bibitem[Liu et al.(2025)]{31}
  Liu, L., Du, D., Sun, Y., \& Li, Y., 2025. SFMW-YOLO: A lightweight
  metal casting surface defect detection method based on modified
  YOLOv8s. Expert Systems with Applications 287. doi:
  10.1016/j.eswa.2025.128170.

  \bibitem[Tsai et al.(2021)]{7}
  Tsai, D. M. \& Jen, P. H. 2021., Autoencoder-based anomaly detection
  for surface defect inspection. Advanced Engineering Informatics 48,
  1-12. doi: 10.1016/j.aei.2021.101272.

  \bibitem[Chen et al.(2021)]{8}
  Chen, Y., Ding, Y., Zhao, F., Zhang, E., Wu, Z. \& Shao, L., 2021.
  Surface defect detection methods for industrial products: A review.
  Applied Sciences 11(16), 1-25. doi: 10.3390/app11167657.

  \bibitem[Liu et al.(2025)]{9}
  Liu, Y., Qiu, W., Fu, K., Chen, X., Wu, L. \& Sun, M., 2025. An
  improved YOLOv8 model and mask convolutional autoencoder for
  multi-scale defect detection of ceramic tiles. Measurement 248,
  1-11. doi: 10.1016/j.measurement.2025.116847.

  \bibitem[Wang et al.(2024)]{10}
  Wang, L., 2024. Robot arm grasping based on YOLOv5 in the
  perspective of automated production. Engineering Research Express,
  6(4), 1-12. doi:10.1088/2631-8695/ad88dc.

  \bibitem[Kato et al.(2022)]{11}
  Kato, H., Nagata, F., Murakami, Y. \& Koya, K., 2022. Partial depth
  estimation with single image using YOLO and CNN for robot arm
  control. IEEE International Conference on Mechatronics and
  Automation (ICMA), 1727-1731. IEEE. doi: 10.1109/ICMA54519.2022.9856055.

  \bibitem[Kim et al.(2021)]{12}
  Kim, M. \& Kim, S. 2021., YOLO-based robotic grasping. International
  Conference on Control, Automation and
  Systems (ICCAS) 21, 1120-1122. doi: 10.23919/ICCAS52745.2021.9649837.

  \bibitem[Jin et al.(2025)]{13}
  Jin, T., Han, X., Wang, P., Zhang, Z., Guo, J. \& Ding, F., 2025.
  Enhanced deep learning model for apple detection, localization, and
  counting in complex orchards for robotic arm-based harvesting.
  Smart Agricultural Technology 10, 1-25. doi: 10.1016/j.atech.2025.100784.

  \bibitem[Bionda et al.(2022)]{14}
  Bionda, A., Frittoli, L. \& Boracchi, G., 2022. Deep
  autoencoders for anomaly detection in textured images using
  CW-SSIM. International Conference on Image Analysis and
  Processing, 669-680. doi: 10.1007/978-3-031-06430-2\_56.

  \bibitem[Yun et al.(2023)]{15}
  Yun, H., Kim, H., Jeong, Y. H. \& Jun, M. B., 2023.
  Autoencoder-based anomaly detection of industrial robot arm using
  stethoscope based internal sound sensor. Journal of Intelligent
  Manufacturing 34(3), 1427-1444. doi: 10.1016/j.ymssp.2004.10.013.

  \bibitem[Jia et al.(2023)]{16}
  Jia, H. \& Liu, W., 2023. Anomaly detection in images with shared
  autoencoders. Frontiers in Neurorobotics 16, 1-11. doi:
  10.3389/fnbot.2022.1046867.

  \bibitem[Kozamernik et al.(2025)]{17}
  Kozamernik, N. \& Bračun, D., 2025. A novel FuseDecode Autoencoder
  for industrial visual inspection: Incremental anomaly detection
  improvement with gradual transition from unsupervised to
  mixed-supervision learning with reduced human effort. Computers in
  Industry 164, 1-19. doi: 10.1016/j.compind.2024.104198.

  \bibitem[Ruediger-Flore et al.(2024)]{18}
  Ruediger-Flore, P., Klar, M., Hussong, M., Mukherjee, A., Glatt,
  M. \& Aurich, J. C., 2024. Comparing Binary Classification and
  Autoencoders for Vision-Based Anomaly Detection in Material Flow.
  Procedia CIRP 121, 138-143. doi: 10.1016/j.procir.2023.09.241.

  \bibitem[Ragab et al.(2024)]{19}
  Ragab, M. G., Abdulkadir, S. J., Muneer, A., Alqushaibi, A.,
  Sumiea, E. H., Qureshi, R. et al., 2024. A
  comprehensive systematic review of YOLO for medical object
  detection (2018 to 2023). IEEE Access 12, 57815-57836. doi:
  10.1109/ACCESS.2024.3386826.

  \bibitem[Hussain(2024)]{20}
  Hussain, M., 2024. Unveiling each variant–a comprehensive review of
  yolo. IEEE access 12, 42816-42833. doi: 10.1109/ACCESS.2024.3378568.

  \bibitem[Padilla et al.(2021)]{21}
  Padilla, R., Passos, W. L., Dias, T. L., Netto, S. L., \& Da Silva,
  E. A., 2021. A comparative analysis of object detection metrics
  with a companion open-source toolkit. Electronics 10(3), 1-28. doi:
  10.3390/electronics10030279.

  \bibitem[Kaur dan Singh(2023)]{22}
  Kaur, R., \& Singh, S., 2023. A comprehensive review of object
  detection with deep learning. Digital Signal Processing 132, 1-17.
  doi: 10.1016/j.dsp.2022.103812.

  \bibitem[Casas et al.(2023)]{23}
  Casas, E., Ramos, L., Bendek, E., \& Rivas-Echeverría, F., 2023.
  Assessing the effectiveness of YOLO architectures for smoke and
  wildfire detection. IEEE Access 11, 96554-96583. doi:
  10.1109/ACCESS.2023.3312217.

  \bibitem[Cao et al.(2024)]{24}
  Cao, L., Shen, Z., \& Xu, S., 2024. Efficient forest fire detection
  based on an improved YOLO model. Visual Intelligence 2(20), 1-7.
  doi: 10.1007/s44267-024-00053-y.

  \bibitem[Mansour et al.(2021)]{25}
  Mansour, R. F., Escorcia-Gutierrez, J., Gamarra, M., Gupta, D.,
  Castillo, O., \& Kumar, S. 2021. Unsupervised deep learning based
  variational autoencoder model for COVID-19 diagnosis and
  classification. Pattern Recognition Letters, 151, 267-274.

  % \bibitem[Kingma dan Welling(2013)]{26}
  % Kingma, D. P., \& Welling, M. 2013. Auto-encoding
  % variational bayes.

  \bibitem[Wei et al.(2020)]{26}
  Wei, R., \& Mahmood, A., 2020. Recent advances in variational
  autoencoders with representation learning for biomedical
  informatics: A survey. Ieee Access 9, 4939-4956. doi:
  0.1109/ACCESS.2020.3048309.

  \bibitem[Najjar(2024)]{27}
  Najjar, Y., A., 2024. Comparative analysis of image quality
  assessment metrics: MSE, PSNR, SSIM and FSIM. International Journal
  of Science and Research 13(3), 110-114. doi: orcid.org/0000-0002-3369-4999.

  \bibitem[Nahm(2022)]{28}
  Nahm, F. S., 2022. Receiver operating characteristic curve:
  overview and practical use for clinicians. Korean journal of
  anesthesiology 75(1), 25-36. doi: 10.4097/kja.21209.

\end{thebibliography}

\addcontentsline{toc}{chapter}{LAMPIRAN}
\chapter*{LAMPIRAN}

\section*{Lampiran 1. \normalfont{Peralatan penelitian}}
\label{Lampiran 1}
\begin{longtable}{c >{\centering\arraybackslash}m{4cm} m{8cm}}
  \toprule
  \textbf{No.} & \textbf{Alat dan Bahan} & \textbf{Fungsi} \\
  \midrule
  \endfirsthead

  \toprule
  \textbf{No.} & \textbf{Alat dan Bahan} & \textbf{Fungsi} \\
  \midrule
  \endhead

  \bottomrule
  \endfoot
  1 &
  \begin{tabular}[c]{@{}c@{}}
    \includegraphics[width=2cm]{gambar/lampiran/arduino.jpg} \\
    Arduino Uno
  \end{tabular}
  & \vspace*{0.5cm} Berfungsi sebagai mikrokontroler utama yang
  mengendalikan motor servo pada lengan robot dan menerima sinyal
  dari sensor. \\
  2 &
  \begin{tabular}[c]{@{}c@{}}
    \includegraphics[width=2cm]{gambar/lampiran/servo.jpg} \\
    Motor \textit{servo}
  \end{tabular}
  & \vspace*{0.5cm} digunakan sebagai aktuator untuk menggerakkan
  bagian-bagian lengan robot sesuai perintah dari arduino. \\
  3 &
  \begin{tabular}[c]{@{}c@{}}
    \includegraphics[width=2cm]{gambar/lampiran/robot.jpg} \\
    Lengan Robot
  \end{tabular}
  & \vspace*{0.5cm} struktur mekanik tempat pemasangan motor servo
  dan berperan sebagai sistem penggerak robotik. \\
  4 &
  \begin{tabular}[c]{@{}c@{}}
    \includegraphics[width=2cm]{gambar/lampiran/psu.jpg} \\
    \textit{Power Supply} 5V
  \end{tabular}
  & \vspace*{0.5cm} Berfungsi menyediakan tegangan stabil untuk
  pengoperasian motor \textit{servo}. \\
  5 &
  \begin{tabular}[c]{@{}c@{}}
    \includegraphics[width=2cm]{gambar/lampiran/pir.jpg} \\
    Sensor PIR
  \end{tabular}
  & \vspace*{0.5cm} Digunakan untuk mendeteksi pergerakan dan
  membantu menghitung jumlah kontainer cacat dan maupun tidak cacat
  yang melintas. \\
  6 &
  \begin{tabular}[c]{@{}c@{}}
    \includegraphics[width=2cm]{gambar/lampiran/kamera.jpg} \\
    Kamera
  \end{tabular}
  & \vspace*{0.5cm} Digunakan untuk mengambil citra kontainer kimia,
  yang akan diproses oleh model deteksi objek (YOLO) dan deteksi
  kecacatan (\textit{autoencoder}). \\
  7 &
  \begin{tabular}[c]{@{}c@{}}
    \includegraphics[width=2cm]{gambar/lampiran/laptop.jpg} \\
    Laptop/Komputer
  \end{tabular}
  & \vspace*{0.5cm} Digunakan untuk memuat program ke Arduino Uno,
  serta menjalankan model deteksi objek berbasis YOLO dan
  \textit{autoencoder} untuk analisis visual. \\
  8 &
  \begin{tabular}[c]{@{}c@{}}
    \includegraphics[width=2cm]{gambar/lampiran/jumper.jpg} \\
    Kabel \textit{Jumper}
  \end{tabular}
  & \vspace*{0.5cm} berfungsi menghubungkan berbagai komponen
  elektronik seperti sensor dan aktuator ke papan rangkaian dan Arduino. \\
  9 &
  \begin{tabular}[c]{@{}c@{}}
    \includegraphics[width=2cm]{gambar/lampiran/breadboard.jpg} \\
    Papan Rangkaian
  \end{tabular}
  & \vspace*{0.5cm}berfungsi untuk menyediakan jalur koneksi antar komponen. \\
  \bottomrule
\end{longtable}

\section*{Lampiran 2. \normalfont{Program untuk latih model YOLO}}
\label{Lampiran 2}
\begin{verbatim}
from ultralytics import YOLO
model = YOLO("yolo12n.pt")
results = model.train(data=config, epochs=50)
inference = model("frame_02318.jpg")
inference[0].show()
\end{verbatim}

\newpage
\section*{Lampiran 3. \normalfont{Program untuk latih model
\textit{autoencoder}}}
\label{Lampiran 3}
\begin{verbatim}
import os
import torch
import torch.nn as nn
import torch.nn.functional as F
import torch.optim as optim
from torch.utils.data import DataLoader
from torchvision import transforms, datasets
import torchvision.transforms as transforms
from PIL import Image
import matplotlib.pyplot as plt
import numpy as np

class CVAE(nn.Module):
    def __init__(self, latent_dim=128):
        super(CVAE, self).__init__()
        self.latent_dim = latent_dim
        self.encoder = nn.Sequential(
            nn.Conv2d(3, 32, kernel_size=4, stride=2, padding=1),
            nn.ReLU(),
            nn.Conv2d(32, 64, kernel_size=4, stride=2, padding=1),
            nn.ReLU(),
            nn.Conv2d(64, 128, kernel_size=4, stride=2, padding=1),
            nn.ReLU(),
            nn.Conv2d(128, 256, kernel_size=4, stride=2, padding=1),
            nn.ReLU(),
            nn.Conv2d(256, 512, kernel_size=4, stride=2, padding=1),
            nn.ReLU()
        )
        self.fc_mu = nn.Linear(512 * 4 * 4, latent_dim)
        self.fc_logvar = nn.Linear(512 * 4 * 4, latent_dim)
        self.decoder_input = nn.Linear(latent_dim, 512 * 4 * 4)
        self.decoder = nn.Sequential(
            nn.ConvTranspose2d(512, 256, kernel_size=4, stride=2, padding=1),
            nn.ReLU(),
            nn.ConvTranspose2d(256, 128, kernel_size=4, stride=2, padding=1),
            nn.ReLU(),
            nn.ConvTranspose2d(128, 64, kernel_size=4, stride=2, padding=1),
            nn.ReLU(),
            nn.ConvTranspose2d(64, 32, kernel_size=4, stride=2, padding=1),
            nn.ReLU(),
            nn.ConvTranspose2d(32, 3, kernel_size=4, stride=2, padding=1),
            nn.Sigmoid()
        )

    def reparameterize(self, mu, logvar):
        """Applies the reparameterization trick to sample from N(mu, var)."""
        std = torch.exp(0.5 * logvar)
        eps = torch.randn_like(std)
        return mu + eps * std

    def forward(self, x):
        enc = self.encoder(x)
        enc = enc.view(x.size(0), -1)
        mu = self.fc_mu(enc)
        logvar = self.fc_logvar(enc)
        z = self.reparameterize(mu, logvar)
        dec_input = self.decoder_input(z)
        dec_input = dec_input.view(x.size(0), 512, 4, 4)
        reconstruction = self.decoder(dec_input)
        return reconstruction, mu, logvar

def loss_function(recon_x, x, mu, logvar):
    """
    Compute the VAE loss as the sum of a reconstruction loss (MSE)
    and the KL divergence loss.
    """
    recon_loss = F.mse_loss(recon_x, x, reduction='sum')
    kl_loss = -0.5 * torch.sum(1 + logvar - mu.pow(2) - logvar.exp())
    return recon_loss + kl_loss

def train(model, dataloader, optimizer, device):
    model.train()
    train_loss = 0
    for batch_idx, (data, _) in enumerate(dataloader):
        data = data.to(device)
        optimizer.zero_grad()
        recon, mu, logvar = model(data)
        loss = loss_function(recon, data, mu, logvar)
        loss.backward()
        train_loss += loss.item()
        optimizer.step()
        if batch_idx % 100 == 0:
            print(f'Batch {batch_idx}/{len(dataloader)} Loss: {loss.item()/len(data):.4f}')
    avg_loss = train_loss / len(dataloader.dataset)
    print(f'====> Average training loss: {avg_loss:.4f}')

def infer_anomaly_with_heatmap(image_path):
    """Runs inference on a single image and highlights anomalies with a heatmap."""

    image = Image.open(image_path).convert("RGB")
    image_tensor = transform(image).unsqueeze(0).to(device)

    with torch.no_grad():
        reconstructed, _, _ = model(image_tensor)

    anomaly_map = torch.abs(image_tensor - reconstructed).mean(dim=1, keepdim=True)

    anomaly_map = anomaly_map.squeeze().cpu().numpy()
    anomaly_map = (anomaly_map - anomaly_map.min()) / (anomaly_map.max() - anomaly_map.min())

    orig_np = image_tensor.squeeze(0).permute(1, 2, 0).cpu().numpy()
    recon_np = reconstructed.squeeze(0).permute(1, 2, 0).cpu().numpy()


    fig, axs = plt.subplots(1, 3, figsize=(12, 4))

    axs[0].imshow(orig_np)
    axs[0].set_title("Original Image")
    axs[0].axis("off")

    axs[1].imshow(recon_np)
    axs[1].set_title("Reconstructed Image")
    axs[1].axis("off")

    axs[2].imshow(orig_np)
    axs[2].imshow(anomaly_map, cmap='jet', alpha=0.5)
    axs[2].set_title("Anomaly Heatmap")
    axs[2].axis("off")

    plt.show()

    mse_loss = torch.mean((reconstructed - image_tensor) ** 2).item()
    print(f"Reconstruction Error (MSE): {mse_loss:.6f}")

    threshold = 0.007183
    is_anomalous = mse_loss > threshold
    print(f"Anomaly Detected: {is_anomalous}")

    return mse_loss, is_anomalous

if __name__ == "__main__":
    device = torch.device("cuda" if torch.cuda.is_available() else "cpu")

    transform = transforms.Compose([
        transforms.Resize((128, 128)),
        transforms.ToTensor(),
    ])

    train_dir = "data_ae/train"

    train_dataset = datasets.ImageFolder(train_dir, transform=transform)

    train_loader = DataLoader(train_dataset, batch_size=64, shuffle=True, num_workers=4)

    model = CVAE(latent_dim=128).to(device)
    optimizer = optim.Adam(model.parameters(), lr=1e-3)
    epochs = 50

    for epoch in range(1, epochs + 1):
        print(f"Epoch {epoch}/{epochs}")
        train(model, train_loader, optimizer, device)

\end{verbatim}

\newpage
\section*{Lampiran 4. \normalfont{Program kombinasi YOLO dan
\textit{autoencoder}}}
\label{Lampiran 4}
\begin{verbatim}
import cv2
import torch
import torch.nn as nn
import torchvision.transforms as transforms
from PIL import Image
from ultralytics import YOLO
import time
import collections
import serial
import os
from datetime import datetime
import matplotlib.pyplot as plt

class CVAE(nn.Module):
    def __init__(self, latent_dim=128):
        super(CVAE, self).__init__()
        self.latent_dim = latent_dim
        self.encoder = nn.Sequential(
            nn.Conv2d(3, 32, kernel_size=4, stride=2, padding=1),
            nn.ReLU(),
            nn.Conv2d(32, 64, kernel_size=4, stride=2, padding=1),
            nn.ReLU(),
            nn.Conv2d(64, 128, kernel_size=4, stride=2, padding=1),
            nn.ReLU(),
            nn.Conv2d(128, 256, kernel_size=4, stride=2, padding=1),
            nn.ReLU(),
            nn.Conv2d(256, 512, kernel_size=4, stride=2, padding=1),
            nn.ReLU(),
        )
        self.fc_mu = nn.Linear(512 * 4 * 4, latent_dim)
        self.fc_logvar = nn.Linear(512 * 4 * 4, latent_dim)
        self.decoder_input = nn.Linear(latent_dim, 512 * 4 * 4)
        self.decoder = nn.Sequential(
            nn.ConvTranspose2d(512, 256, kernel_size=4, stride=2, padding=1),
            nn.ReLU(),
            nn.ConvTranspose2d(256, 128, kernel_size=4, stride=2, padding=1),
            nn.ReLU(),
            nn.ConvTranspose2d(128, 64, kernel_size=4, stride=2, padding=1),
            nn.ReLU(),
            nn.ConvTranspose2d(64, 32, kernel_size=4, stride=2, padding=1),
            nn.ReLU(),
            nn.ConvTranspose2d(32, 3, kernel_size=4, stride=2, padding=1),
            nn.Sigmoid(),
        )

    def reparameterize(self, mu, logvar):
        std = torch.exp(0.5 * logvar)
        eps = torch.randn_like(std)
        return mu + eps * std

    def forward(self, x):
        enc = self.encoder(x)
        enc = enc.view(x.size(0), -1)
        mu = self.fc_mu(enc)
        logvar = self.fc_logvar(enc)
        z = self.reparameterize(mu, logvar)
        dec_input = self.decoder_input(z)
        dec_input = dec_input.view(x.size(0), 512, 4, 4)
        reconstruction = self.decoder(dec_input)
        return reconstruction, mu, logvar

def get_centroid(x1, y1, x2, y2):
    return ((x1 + x2) // 2, (y1 + y2) // 2)

def check_anomaly(crop_img):
    image = Image.fromarray(cv2.cvtColor(crop_img, cv2.COLOR_BGR2RGB))
    image_tensor = transform(image).unsqueeze(0).to(device)
    with torch.no_grad():
        reconstructed, _, _ = ae_model(image_tensor)
    mse_loss = torch.mean((reconstructed - image_tensor) ** 2).item()
    return mse_loss

def save_anomaly_heatmap(image_tensor, reconstructed, loss_value, obj_id):
    anomaly_map = torch.abs(image_tensor - reconstructed).mean(dim=1, keepdim=True)
    anomaly_map = anomaly_map.squeeze().cpu().numpy()
    anomaly_map = (anomaly_map - anomaly_map.min()) / (
        anomaly_map.max() - anomaly_map.min() + 1e-8
    )

    orig_np = image_tensor.squeeze(0).permute(1, 2, 0).cpu().numpy()
    recon_np = reconstructed.squeeze(0).permute(1, 2, 0).cpu().numpy()

    fig, axs = plt.subplots(1, 3, figsize=(12, 4))

    axs[0].imshow(orig_np)
    axs[0].set_title("Original Image")
    axs[0].axis("off")

    axs[1].imshow(recon_np)
    axs[1].set_title("Reconstructed Image")
    axs[1].axis("off")

    axs[2].imshow(orig_np)
    axs[2].imshow(anomaly_map, cmap="jet", alpha=0.5)
    axs[2].set_title(f"Anomaly Heatmap\nLoss: {loss_value:.6f}")
    axs[2].axis("off")

    os.makedirs("output", exist_ok=True)
    timestamp = datetime.now().strftime("%Y%m%d_%H%M%S")
    filename = f"output/heatmap_{obj_id}_{timestamp}.png"
    plt.savefig(filename)
    plt.close(fig)

    print(f"[Saved] Heatmap image saved to {filename}")

TRANSFORM = transforms.Compose([transforms.Resize((128, 128)), transforms.ToTensor()])
DEVICE = torch.device("cuda" if torch.cuda.is_available() else "cpu")
AE_MODEL_PATH = "ae.pth"
YOLO_MODEL_PATH = "./runs/detect/train/weights/best.pt"
CONFIDENCE_THRESHOLD = 0.90
LOSS_BATCH_SIZE = 20
DELAY_SECONDS = 0.005
OBJECT_WARMUP_FRAMES = 5
CENTROID_TOLERANCE = 30
SPIKE_FILTER_THRESHOLD = 0.0099
SERIAL_PORT = "/dev/ttyACM0"
BAUD_RATE = 9600
DEFECT_THRESHOLD = 0.007183

print("Loading models...")
device = DEVICE
transform = TRANSFORM
ae_model = CVAE(latent_dim=128).to(device)
ae_model.load_state_dict(torch.load(AE_MODEL_PATH, map_location=device))
ae_model.eval()
yolo_model = YOLO(YOLO_MODEL_PATH)
print(f"Models loaded successfully on {device}.")

print("Initializing camera...")
cap = cv2.VideoCapture(0)
if not cap.isOpened():
    print("Error: Cannot open camera")
    exit()
print("Camera initialized.")

arduino = None
try:
    print(f"Connecting to Arduino on {SERIAL_PORT}...")
    arduino = serial.Serial(SERIAL_PORT, BAUD_RATE, timeout=1)
    time.sleep(2)
    print("Arduino connected.")
except serial.SerialException as e:
    print(f"WARNING: Could not connect to Arduino. {e}")
    print("--> Running in SIMULATION MODE. No data will be sent.")

object_history = collections.defaultdict(dict)

print("\nStarting monitoring... Press Ctrl+C to exit.")
try:
    while True:
        ret, frame = cap.read()
        if not ret:
            print("Failed to grab frame. Exiting.")
            break

        results = yolo_model(frame, verbose=False)[0]
        current_time = time.time()

        for box in results.boxes:
            if box.conf[0] < CONFIDENCE_THRESHOLD:
                continue

            x1, y1, x2, y2 = map(int, box.xyxy[0])
            crop = frame[y1:y2, x1:x2]
            centroid = get_centroid(x1, y1, x2, y2)

            for obj_id in list(object_history.keys()):
                if (
                    current_time - object_history[obj_id].get("last_seen", current_time)
                    > 2
                ):
                    print(f"[Cleanup] Removing stale object ID {obj_id}")
                    del object_history[obj_id]

            matched_id = None
            for obj_id, data in object_history.items():
                prev_centroid = data.get("last_centroid")
                if (
                    prev_centroid
                    and abs(prev_centroid[0] - centroid[0]) < CENTROID_TOLERANCE
                    and abs(prev_centroid[1] - centroid[1]) < CENTROID_TOLERANCE
                ):
                    data["seen_frames"] += 1
                    data["last_centroid"] = centroid
                    data["last_seen"] = current_time
                    matched_id = obj_id
                    break

            if matched_id is None:
                matched_id = len(object_history)
                object_history[matched_id] = {
                    "seen_frames": 1,
                    "last_centroid": centroid,
                    "valid_losses": [],
                    "first_sent": False,
                    "last_seen": current_time,
                }
                print(f"[New Object] Assigned ID {matched_id}")

            current_obj = object_history[matched_id]

            if current_obj["seen_frames"] <= OBJECT_WARMUP_FRAMES:
                print(
                    f"[Warmup] Skipping object {matched_id}, frame {current_obj['seen_frames']}"
                )
                continue

            anomaly_score = check_anomaly(crop)

            if anomaly_score > SPIKE_FILTER_THRESHOLD:
                print(f"[Spike Filter] Score {anomaly_score:.6f} too high, ignored.")
                current_obj["valid_losses"].clear()
                continue

            current_obj["valid_losses"].append(anomaly_score)

            if len(current_obj["valid_losses"]) >= LOSS_BATCH_SIZE:
                avg_error = sum(current_obj["valid_losses"]) / len(
                    current_obj["valid_losses"]
                )

                if not current_obj["first_sent"]:
                    print(
                        f"[Skip First Send] Skipping first Arduino send for object {matched_id}"
                    )
                    current_obj["first_sent"] = True
                    current_obj["valid_losses"].clear()
                    continue

                image_tensor = (
                    transform(Image.fromarray(cv2.cvtColor(crop, cv2.COLOR_BGR2RGB)))
                    .unsqueeze(0)
                    .to(device)
                )
                with torch.no_grad():
                    reconstructed, _, _ = ae_model(image_tensor)

                save_anomaly_heatmap(image_tensor, reconstructed, avg_error, matched_id)

                print(f"[*] Object {matched_id} final avg loss: {avg_error:.6f}")
                label = "DEFECT" if avg_error > DEFECT_THRESHOLD else "OK"
                msg = f"{label},{avg_error:.6f}\n"

                if arduino:
                    try:
                        print(f"--> Sending to Arduino: {msg.strip()}")
                        arduino.write(msg.encode())
                    except serial.SerialException as e:
                        print(f"ERROR: Arduino write failed: {e}. Disconnecting.")
                        arduino.close()
                        arduino = None
                else:
                    print(f"--> [SIMULATED] Arduino message: {msg.strip()}")

                current_obj["valid_losses"].clear()
            else:
                print(
                    f"[Collecting] Object {matched_id} sample {len(current_obj['valid_losses'])}/{LOSS_BATCH_SIZE}"
                )

        time.sleep(DELAY_SECONDS)

except KeyboardInterrupt:
    print("\nInterruption detected. Shutting down.")
finally:
    cap.release()
    print("Camera released.")
    if arduino and arduino.is_open:
        arduino.close()
        print("Arduino connection closed.")
    print("Exited gracefully.")
\end{verbatim}

\newpage
\section*{Lampiran 5. \normalfont{Program lengan robot}}
\label{Lampiran 5}
\begin{verbatim}
#include <Servo.h>
Servo servo2;
Servo servo3;
Servo servo4;

void write_servo(int from, int to, int delay_rotate, Servo &servo) {
  if (from < to) {
    for (int pos = from; pos <= to; pos++) {
      servo.write(pos);
      delay(delay_rotate);
    }
  } else {
    for (int pos = from; pos >= to; pos--) {
      servo.write(pos);
      delay(delay_rotate);
    }
  }
}

void setup() {
  servo2.attach(5);
  servo3.attach(6);
  servo4.attach(9);

  servo2.write(85);
  servo3.write(30);
  servo4.write(170);
  delay(2000);

  Serial.begin(9600);
}

void loop() {
  if (Serial.available()) {
    String msg = Serial.readStringUntil('\n');

    int sepIndex = msg.indexOf(',');
    if (sepIndex > 0) {
      String label = msg.substring(0, sepIndex);
      String errorStr = msg.substring(sepIndex + 1);
      float errorVal = errorStr.toFloat();

      if (label == "DEFECT" || label == "OK") {
        Serial.print(label);
        Serial.print(",");
        Serial.println(errorVal, 5);

        if (label == "DEFECT") {
          // KIRI
          write_servo(50, 120, 15, servo3);
          delay(2000);

          write_servo(170, 30, 5, servo4);
          delay(2000);

          write_servo(120, 50, 15, servo3);
          delay(2000);

          write_servo(85, 180, 30, servo2);
          delay(2000);

          write_servo(30, 170, 5, servo4);
          delay(2000);

          write_servo(180, 85, 30, servo2);
          delay(2000);

        } else if (label == "OK") {
          // KANAN
          write_servo(50, 120, 15, servo3);
          delay(2000);

          write_servo(170, 30, 5, servo4);
          delay(2000);

          write_servo(120, 50, 15, servo3);
          delay(2000);

          write_servo(85, 0, 30, servo2);
          delay(2000);

          write_servo(30, 170, 5, servo4);
          delay(2000);

          write_servo(0, 85, 30, servo2);
          delay(2000);
        }
      }
    }
  }
}
\end{verbatim}

\vspace{1em}

\section*{Lampiran 6. \normalfont{Dokumentasi pergerakan lengan robot}}
\label{Lampiran 6}
\begin{figure}[H]
  \centering
  \includegraphics[width=0.8\textwidth]{gambar/lampiran/robot.png} \\
  Lengan robot menunggu hasil deteksi cacat
\end{figure}
\vspace{-1em}

\begin{figure}[H]
  \centering
  \includegraphics[width=0.8\textwidth]{gambar/robot_normal.jpeg} \\
  Lengan robot ketika meyortir kontainer normal
\end{figure}
\vspace{-1em}

\begin{figure}[H]
  \centering
  \includegraphics[width=0.8\textwidth]{gambar/robot_cacat.jpeg} \\
  Lengan robot ketika meyortir kontainer cacat
\end{figure}
\vspace{-1em}

\end{document}
